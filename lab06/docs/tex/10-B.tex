\chapter*{\hypertarget{apB}{}Приложение Б\\Параметризация для класса данных 2}
\addcontentsline{toc}{chapter}{Приложение Б Параметризация для класса данных 2}

\renewcommand{\thetable}{\textmd{Б.1}}

\noindent
\begin{center}
    \captionsetup{format=hang,justification=raggedright,
                  singlelinecheck=off,width=8.4cm}
    \begin{longtable}[c]{|r|r|r|r|}
        \caption{Параметризация~~~~~~для класса~~~~~~~данных~~~~~~~с
                 большим~~~~~~~разбросом значений}
        \\\hline
        $\alpha$ & $p$ & Число итераций & Ошибка \\
        \hline
        \endfirsthead
        \captionsetup{labelsep=none}
        \caption[]{ (продолжение)}\\
        \hline
        $\alpha$ & $p$ & Число итераций & Ошибка \\
        \endhead
        \csvreader[
            late after line=\\\hline,
            late after last line=,
            before reading={\catcode`\#=12},
            after reading={\catcode`\#=6}
        ]
        {../data/csv/highParam.csv}{1=\colo,2=\coltw,3=\colt,4=\colf}
        {\colo & \coltw & \colt & \colf}
        \\\hline
    \end{longtable}
\end{center}
