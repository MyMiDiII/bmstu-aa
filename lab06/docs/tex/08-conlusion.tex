\chapter*{Заключение}
\addcontentsline{toc}{chapter}{Заключение}

В ходе исследования был проведен сравнительный алгорима полного перебора и 
муравьного алгоритма для решения задачи коммивояжера. При малых значениях
количества вершин в графе до 8 следует использовать алгоритм полного перебора,
при больших значениях следует использовать муравьиный алгоритм, который дает
точный результат при различных разбросах стоимостей  при паре значений
параметров $(\alpha, p)$ равной $(0.5, 0.5)$ и количестве итераций от 50.

В ходе выполения лабораторной работы:
\begin{itemize}[left=\parindent]
    \item были описаны и разработаны алгоритм полного перебора и муравьиный
        алгоритм для решения задачи коммивояжера;
    \item был реализован каждый из описанных алгоритмов;
    \item было проведено тестирование каждого из алгоритмов;
    \item по экспериментальным данным были сделаны выводы об эффективности по
          времени каждого из реализованных алгоритмов;
    \item была проведена параметризация муравьиного алгоритма на двух класса
        данных и сделаны выводы по её результатам.
\end{itemize}

Таким образом, все поставленные задачи были выполнены, а цель достигнута.
