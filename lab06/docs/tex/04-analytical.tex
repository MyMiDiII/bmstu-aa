\chapter{Аналитическая часть}

В данном разделе представлено теоретическое описание задачи коммивояжера, а
алгоритм полного перебора и муравьиный алгоритм для её решения.

\section{Задача коммивояжера}

\textbf{Задача коммивояжера} (или \textit{задача о путешествующем торговце})
состоит в поиске кратчайшего замкнутого маршрута, проходящего через все города
ровно 1 раз \cite{Shtovba}. Если говорить формально, то необходимо найти
гамильтонов цикл минимального веса во взвешенном полном графе.

\section{Алгоритм полного перебора}

\textbf{Алгоритм полного перебора} для решения задачи коммивояжера предполагает
рассмотрение всех возможных путей в графе и выбор наименьшего из них. Данный
алгоритм имеет высокую сложность $O(n!)$, что большие затраты по времени даже
при небольших значениях числа вершин в графе.

\section{Муравьиный алгоритм}

Муравьиные алгоритмы представляют собой новый перспективный метод решения задач
оптимизации, в основе которого лежит моделирование поведения колонии муравьев
\cite{Ulianov}. Колония представляет собой систему с очень простыми правилами
автономного поведения особей.

Каждый муравей определяет для себя маршрут, который необходимо пройти на основе
феромона, который он ощущает, во время прохождения, каждый муравей оставляет
феромон на своем пути, чтобы остальные муравьи могли по нему ориентироваться.
При этом феромон испаряется для исключения случая, когда все муравьи движутся
одним и тем же субоптимальным маршрутом. В результате при прохождении каждым
муравьем различного маршрута наибольшее число феромона остается на оптимальном
пути.

Для каждого муравья переход из города $i$ в город $j$ зависит от трех
составляющих:
\begin{itemize}
    \item Память муравья (tabu list) — это список посещенных муравьем городов,
        заходить в которые еще раз нельзя. Используя этот список, муравей
        гарантированно не попадет в один и тот же город дважды.
    \item Видимость — величина, обратная расстоянию: $\eta_{ij} = 1 / D_{ij}$,
        где $D_{ij}$ — расстояние между городами $i$ и $j$. Эта величина
        выражает эвристическое желание муравья поседить город $i$ из города
        $j$, чем ближе город, тем больше желание его посетить.
    \item Виртуальный след феромона $\tau_{ij}$ на ребре $(i, j)$ представляет
        подтвержденное муравьиным опытом желание посетить город $j$ из города
        $i$.
\end{itemize}

Вероятность перехода $k$-ого муравья из города $i$ в город $j$ на $t$-й
итерации определяется формулой \ref{eq:1}:

\begin{equation}\label{eq:1}
      \begin{cases}
          P_{ij,k}(t) = \frac{[\tau_{ij}(t)]^{\alpha} \cdot
                        [\eta_{ij}]^{\beta}}{\displaystyle\sum_{l \in
                        J_{i,k}}[\tau_{il}(t)]^{\alpha} \cdot
                        [\eta_{il}]^{\beta}} & \quad \text{если } j \in
                        J_{i,k},\\
          P_{ij,k}(t) = 0 & \quad \text{если } j \notin J_{i,k}
      \end{cases}
      ,
\end{equation}

где $\alpha, \beta$ -- настраиваемые параметры, $J_{i,k}$ - список городов,
которые надо посетить $k$-ому муравью, находящемуся в $i$-ом городе, $\tau$ -
концентрация феромона, а при $\alpha = 0$ алгоритм вырождается в жадный.

После завершения движения всеми муравьями происходит обновление феромона.
Если $p \in [0, 1]$ -- коэффициент испарения феромона, то новое значения
феромона на ребре $(i,j)$ рассчитывается по формуле \ref{eq:2}:
\begin{equation}\label{eq:2}
    \tau_{ij}(t+1) = (1-p)\tau_{ij}(t) + \Delta \tau_{ij},~~\Delta \tau_{ij} =
                     \displaystyle\sum_{k=1}^N \tau^k_{ij}
\end{equation}
где
\begin{equation}\label{eq:3}
    \Delta \tau^k_{ij} = \begin{cases}
        \frac{Q}{L_k}, & \quad \textrm{ребро посещено k-ым муравьем,} \\
        0, & \quad \textrm{иначе}
    \end{cases}
\end{equation}
$L_{k}$ — длина пути k-ого муравья, $Q$ — некоторая константа порядка длины
путей, $N$ — количество муравьев \cite{Shtovba}.
 
\section{Вывод}

В данном разделе была описана задача коммивояжера и алгоритмы её решения:
полный перебор и муравьиный алгортим. Из представленных описаний можно
предъявить ряд требований к разрабатываемому программному обеспечению:
\begin{itemize}[left=\parindent]
    \item на вход должна подаваться матрица смежности графа, представляющего
        города и стоимость переезда между ними;
    \item для муравьиного алгоритма также должны подаваться на вход
        коэффициенты $\alpha, p$ и количество итераций;
    \item на выходе должны выдаваться искомый путь и его длина;
    \item при неверных входных данных должна выдаваться ошибка.
\end{itemize}
