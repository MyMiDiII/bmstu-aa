\chapter*{Введение}
\addcontentsline{toc}{chapter}{Введение}

В последние два десятилетия при оптимизации сложных систем исследователи все
чаще применяют природные механизмы поиска наилучших решений. Одним из таких
механизмов являются муравьиные алгоритмы -- новый перспективный метод
оптимизации, базирующийся на моделировании поведения колонии муравьев
\cite{Shtovba}. Первой задачей, к которой был применен муравьиный алгоритм,
является проблема коммивояжера (или Travelling Salesman Problep -- TSP)
\cite{Zaychenko}. На основе этой задачи будет проводиться исследование
муравьиного алгоритма в данной лабораторной работе.

\textbf{Целью данной работы} является проведение сравнительного анализа метода
полного перебора и эвристического метода на базе муравьиного алгоритма.

Для достижения поставленной цели необходимо выполнить следующие
\textbf{задачи}:
\begin{itemize}[left=\parindent]
    \item описать алгоритм полного перебора для решения задачи коммивояжера;
    \item описать муравьиный алгоритм для решения задачи коммивояжера; 
    \item описать функциональные требования к программе;
    \item разработать описанные алгоритмы;
    \item реализовать алгоритм полного перебора и муравьиный алгоритм для
        решения задачи коммивояжера;
    \item провести тестирование реализованных алгоритмов;
    \item провести сравнительный анализ алгоритмов по времени работы
          реализаций;
    \item провести параметризацию муравьиног алгоритма на двух классах данных;
    \item сделать выводы по полученным результатам.
\end{itemize}
