\chapter{Исследовательская часть}

\section{Технические характеристики}

Технические характеристики устройства, на котором выполнялось тестирование:

\begin{itemize}
	\item Операционная система: Manjaro \cite{manjaro} Linux x86\_64.
	\item Память: 8 GiB.
    \item Процессор: Intel® Core™ i5-8265U, 4 физических ядра, 8 логических
        ядра\cite{intel}.
\end{itemize}

Тестирование проводилось на ноутбуке, включенном в сеть электропитания. Во
время тестирования ноутбук был нагружен только встроенными приложениями
окружения, окружением, а также непосредственно системой тестирования.

\section{Примеры работы программы}

На рисунке \ref{img:exp} представлены результаты работы программы.

\img{100mm}{exp}{Пример работы программы}{exp}

\section{Результаты тестирования}

Программа была протестирована на входных данных, приведенных в таблице
\ref{tab:tests}. Полученные результаты работы программы совпали с ожидаемыми
результатами.

\section[Постановка эксперимента по замеру времени]
        {Постановка ~~эксперимента ~~по ~~замеру времени}

Для оценки времени работы алгоритма полного перебора и муравьиного алгоритма
был проведен эксперимент, в котором определялось влияние количества вершин в
графе на время работы каждого из алгоритмов. Тестирование проводилось на
количестве вершин от 2 до 10 с шагом 1. Так как от запуска к запуску
процессорное время, затрачиваемое на выполнение алгоритмов, менялось в
определенном промежутке, необходимо было усреднить вычисляемые значения. Для
этого каждый алгоритм на каждом значении запускался по 10 раз, и для полученных
10 значений определялось среднее арифметическе, которое заносилось в таблицу.

Также был проведен эксперимент для определения оптимальных параметров
муравьиного алгоритма. Для этого алгоритм запускался на всех сочетаниях
значений параметров, где коэффициенты $\alpha$ и $p$ изменились в промежутке от
0 до 1 с шагом 0.1, а количество итераций принимало занчения 1, 5, 10, 50, 100,
500, 100; и полученный результат сравнивался с точным результатом, полученным
с помощью алгоритма полного перебора. Параметризация проводилась на двух
различных классах данных с малым и большим разбросом значений стоимостей
перехода из одного города в другой. Размер матриц, на которых проводилась
параметризация был выбран равный 9, как наибольший размер, для которого
алгоритм полного перебора решает задачу за малое время.

Результаты эксперимента были представлены в виде таблиц и графиков, приведенных
в следующем подразделе.

\section{Результаты эксперимента}

В таблице \ref{tab:times} представлены результаты измерения времени работы
алгоритмов решения задачи коммивояжера от количества вершин в графе.
На рисунке \ref{img:graph} представлен соответствующий график.

В таблицах \ref{tab:low}-\ref{tab:high} представлена часть результатов
параметризации муравьиного алгоритма на классе данных 1 с малым разбросом
значений и классе данных 2 с большим разбросом значений соответственно. Полный
таблицы результатов представлены в приложениях А и Б соответственно.

\begin{table}[h]
    \begin{center}
    \begin{threeparttable}
        \captionsetup{format=hang,justification=raggedright,
                      singlelinecheck=off}
        \caption{\label{tab:times}Время работы от количества вершин}
        \begin{tabular}{|r|r|r|}
            \hline
            \bfseries Количество врешин & \bfseries Перебор, нс
            & \bfseries Муравьиный, нс
            \csvreader{../data/csv/times.csv}{}
            {\\\hline \csvcoli&\csvcolii&\csvcoliii}
            \\\hline
        \end{tabular}
    \end{threeparttable}
    \end{center}
\end{table} 

\img{80mm}{graph}{График зависимости времени работы от числа заявок}{graph}

%\begin{table}[h]
%    \begin{center}
%    \begin{threeparttable}
%        \captionsetup{format=hang,justification=raggedright,
%                      singlelinecheck=off}
%        \caption{\label{tab:byWords}Время работы от количества слов}
%        \begin{tabular}{|r|r|r|}
%            \hline
%            \bfseries Точность & \bfseries Последовательная, нс
%            & \bfseries Параллельная, нс
%            \csvreader{../data/csv/byWords.csv}{}
%            {\\\hline \csvcoli&\csvcolii&\csvcoliii}
%            \\\hline
%        \end{tabular}
%    \end{threeparttable}
%    \end{center}
%\end{table} 
%
%\img{80mm}{byWords}{График зависимости времени работы реализаций от
%количества слов в сообщениях}{byWords}

\clearpage
\section{Вывод}

По результатам эксперимента можно сделать следующие выводы:
\begin{itemize}[left=\parindent]
    \item при количестве заявок до 10 последовательная и конвейерная
         реализация генерации зашифрованных сообщений отрабатывают за
         одинаковое время, а при количестве заявок до 5 последовательная
         реализация работает $1.3$ раза быстрее, что объясняется затратами
         на передачу данных между лентами с помощью очередей/каналов
         в конвейерной реализации;
    \item при большем количестве заявок от 25 конвейерная реализация работает в
        $1.7$ раза быстрее последовательной;
    \item при фиксированном количестве заявок при различных количествах слов в
        сообщениях конвейерная реализация работает в $1.7$ раза быстрее
        последовательной.
\end{itemize}

Таким образом, для обработки количества заявок большего 10 для достижения
оптимальной скорости вычислений необходимо использовать конвейерную реализацию.
Если работа просходит с количеством заявок до 10 достаточно последовательной
реализации, то есть нет необходимости реализовывать более сложный конвейерный
алгоритм.
