\chapter{Конструкторская часть}

В данном разделе разрабатываются алгоритмы решения задачи коммивояжера: полный
перебор и муравьиный алгоритм, также описывается структура программы и способы
её тестироваия.

\section{Разработка алгоритмов}

На рисунке \ref{scheme:brute} представлена схема алгоритма полного перебора для
решения задачи коммивояжера.

На рисунке \ref{scheme:ants} представлена схема муравьиного алгоритма для
решения задачи коммивояжера.

\noindent
\scheme{90mm}{bruteForce}{Схема алгоритма полного перебора}{brute}
\noindent
\scheme{110mm}{ants}{Схема муравьиного алгоритма}{ants}

\clearpage
\section{Структура разрабатываемого ПО}

Для реализации разрабатываемого программного обеспечения будет использоваться
метод структурного программирования. Каждый из алгоритмов будет представлен
отдельной функцией, при необходимости будут выделены подпрограммы для каждой из
них. Также будут реализованы функции для ввода-вывода и функция, вызывающая все
подпрограммы для связности и полноценности программы.

\section{Классы эквивалентности при тестировании}

Для тестирования программного обеспечения во множестве тестов будут выделены
следующие классы эквивалентности:
\begin{itemize}[left=\parindent]
    \item неверно выбран файл с матрицей смежности; 
    \item неверный ввод коэффициента $\alpha$;
    \item неверный ввод коэффициента $p$;
    \item неверный ввод количество интераций;
    \item корректные данные.
\end{itemize}

\section{Вывод}

В данном разделе были разработаны алгоритмы решения задачи коммивояжера, была
описана структура разрабатываемого ПО. Для дальнейшей проверки правильности
работы программы были выделены классы эквивалентности тестов.
