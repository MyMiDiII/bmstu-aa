\chapter{Конструкторская часть}

В данном разделе разрабатываются алгоритмы генерации сообщений, перестановки
букв в словах сообщений и шифра Веженера, алгоритмы линейной обработки и
каждого потока конвейерной обрабокти, включая главный, также описывается
структура программы и способы её тестироваия.

\section{Разработка алгоритмов}

На рисунке \ref{scheme:gen} представлена схема алгоритма генерации сообщений,
на рисунках \ref{scheme:revWord}-\ref{scheme:revMsg} -- схема алгоритма
перестановки букв в словах, на рисунке \ref{scheme:veg} -- схема алгоритма
применения шифра Веженера. 

На рисунке \ref{scheme:lin} представлена схема алгоритма линейной обработки
заявок. На рисунке \ref{scheme:main} представлена схема главного потока,
запускающего потоки этапов конвейера, схемы алгоритмов которых представлены на
рисунках \ref{scheme:first}-\ref{scheme:third}.

%\noindent
%\scheme{110mm}{gen}{Схема алгоритма генерации сообщений}{gen}
%\noindent
%\scheme{70mm}{revWord}{Схема алгоритма перестановки символов в обратном порядке
%в строке}{revWord}
%\noindent
%\scheme{95mm}{revMsg}{Схема алгоритма перестановки букв в словах
%сообщения}{revMsg}
%\noindent
%\scheme{110mm}{veg}{Схема алгоритма применения шифра Веженера}{veg}
%\noindent
%\scheme{90mm}{lin}{Схема линейной обработки заявок}{lin}
%\noindent
%\scheme{90mm}{main}{Схема главного потока конвейерной обработки заявок}{main}
%\noindent
%\scheme{90mm}{first}{Схема потока первого этапа обработки}{first}
%\noindent
%\scheme{90mm}{second}{Схема потока второго этапа обработки}{second}
%\noindent
%\scheme{90mm}{third}{Схема потока третьего этапа обработки}{third}

\clearpage
\section{Структура разрабатываемого ПО}

Для реализации разрабатываемого программного обеспечения будет использоваться
метод структурного программирования. Каждый из алгоритмов будет представлен
отдельной функцией, при необходимости будут выделены подпрограммы для каждой из
них. Также будут реализованы функции для ввода-вывода и функция, вызывающая все
подпрограммы для связности и полноценности программы.

\section{Классы эквивалентности при тестировании}

Для тестирования программного обеспечения во множестве тестов будут выделены
следующие классы эквивалентности:
\begin{itemize}[left=\parindent]
    \item отрицательное число;
    \item ноль заявок; 
    \item нечисловые данные;
    \item разные порядки верного количества заявок.
\end{itemize}

\section{Вывод}

В данном разделе были разработаны алгоритмы этапов обработки, а также алгоритмы
линейной и конвейерной обработки заявок, была описана структура
разрабатываемого ПО.  Для дальнейшей проверки правильности работы программы
были выделены классы эквивалентности тестов.
