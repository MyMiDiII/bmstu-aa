\chapter{Технологическая часть}

В данном разделе описаны требования к программному обеспечению, средства
реализации, приведены листинги кода и данные, на которых будет проводиться
тестирование.

\section{Требования к программному обеспечению}

Программа должна предоставлять следующие возможности:
\begin{itemize}[left=\parindent]
    \item ввода имени файла словаря;
    \item ввода искомого ключа;
    \item вывода значения по искомому ключу или сообщения, что такого ключа
          нет;
    \item получения времени поиска каждого ключа каждым алгоритмом;
    \item получения количества сравнений при поиске каждого ключа каждым
          алгоритмом.
\end{itemize}

\section{Средства реализации}

Для реализации данной лабораторной работы выбран интерпретируемый язык
программирования высокого уровня Python\cite{python}, так как он позволяет
реализовывать сложные задачи за кратчайшие сроки за счет простоты синтаксиса и
наличия большого количества подключаемых библиотек. 

В качестве среды разработки выбран текстовый редактор Vim\cite{vim} c
установленными плагинами автодополнения и поиска ошибок в процессе написания,
так как он реализует быстрое перемещение по тексту программы и простое
взаимодействие с командной строкой.

Замеры времени проводились при помощи функции process\_time\_ns из библиотеки
time\cite{time}.

\newpage
\section{Листинги кода}

В данном подразделе представлены листинги кода алгоритмов:
\begin{itemize}[]
    \item поиск полным перебором (листинг \ref{lst:bf});
    \item бинарный поиск (листинги \ref{lst:sort}-\ref{lst:bin});
    \item частотный анализ (листинги \ref{lst:gseg}-\ref{lst:binseg}).
\end{itemize}

\mylisting{Поиск полным перебором}{bf}{73-82}{dictionary.py}

\mylisting{Сортировка для бинарного поиска}{sort}{39-43}{dictionary.py}

\mybreaklisting{Бинарный поиск}{bin}{84-102}{dictionary.py}

\mylisting{Разбиение словаря на сегменты}{gseg}{45-65}{dictionary.py}

\mylisting{Поиск в словаре, разбитом на сегменты}{seg}{127-141}{dictionary.py}

\mylisting{Бинарный поиск в сегменте}{binseg}{104-124}{dictionary.py}

\section{Описание тестирования}

В таблице \ref{tab:tests} приведены функциональные тесты программы.

\begin{table}[h!]
	\begin{center}
    \begin{threeparttable}
        \captionsetup{justification=raggedright,singlelinecheck=off}
        \caption{\label{tab:tests}Функциональные тесты}
        \begin{tabular}{|c|c|}
			\hline
            \textbf{Ключ} & \textbf{Ожидаемый результат} \\ [2mm]
            \hline
            j
            &
            Нет такой игры!
            \\
            \hline
            Munchkin
            &
            \{2001, 41605, 5.9\}
            \\
            \hline
            1001
            &
            \{1900, 31, 6.43\}
            \\
            \hline
		\end{tabular}
    \end{threeparttable} 
	\end{center}
\end{table}

\section{Вывод}

В данном разделе были реализованы алгоритмы поиска в словаре: перебором,
бинарный и частотный анализ. Также были написаны тесты для каждого класса
эквивалентности, описанного в конструкторском разделе.
