\chapter*{Заключение}
\addcontentsline{toc}{chapter}{Заключение}

В ходе исследования был проведен сравнительный анализ алгоритмов поиска в
словаре: линейного, бинарного и частнотного анализа. В результате исследования
было выяснено, что самым быстрым алгоритмом является метод частного анализа,
однако он, как и бинарный поиск требует, дополнительных затрат на предоработку,
что дает преимущество линейному поиску при небольших (до 50 единиц) наборах
данных. Бинарный же поиск необходимо использовать в случае, если требуется
сократить время на предварительную обработку данных.

В ходе выполения лабораторной работы:
\begin{itemize}[left=\parindent]
    \item были описаны и разработаны алгоритмы поиска в словаре: линейный,
        бинарный и частный анализ;
    \item был реализован каждый из описанных алгоритмов;
    \item было проведено тестирование реализованных алгоритмов;
    \item были получены зависимости времени работы алгоритмов от положения
        ключа в словаре;
    \item был получены диаграммы количества сравнений от положения ключа в
        словаре;
    \item на основе полученных результатов экспериментов был проведен
        сравнительный анализ алгоритмов и сделаны соответствующие выводы.
\end{itemize}

Таким образом, все поставленные задачи были выполнены, а цель достигнута.
