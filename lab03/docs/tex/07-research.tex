\chapter{Исследовательская часть}

\section{Технические характеристики}

Технические характеристики устройства, на котором выполнялось тестирование:

\begin{itemize}
	\item Операционная система: Manjaro \cite{manjaro} Linux x86\_64.
	\item Память: 8 GiB.
	\item Процессор: Intel® Core™ i5-8265U\cite{intel}.
\end{itemize}

Тестирование проводилось на ноутбуке, включенном в сеть электропитания. Во
время тестирования ноутбук был нагружен только встроенными приложениями
окружения, окружением, а также непосредственно системой тестирования.

\section{Примеры работы программы}

На рисунке \ref{img:exp} представлены результаты работы программы на случайной
последовательности.

\img{60mm}{img01}{Пример работы программы}{exp}

\section{Результаты тестирования}

Программа была протестирования на входных данных, приведенных в таблице
\ref{tab:tests}. Полученные результаты работы программы совпали с ожидаемыми
результатами.

\section[Постановка эксперимента по замеру времени]
        {Постановка ~~эксперимента ~~по ~~замеру времени}

Для оценки времени работы реализаций алгоритмов сортировки был проведен эксперимент, в котором определялось влияние размеров последовательности на время работы каждого из алгоритмов. Тестирование проводилось на размерах последовательностей от 100 до 1000 с шагом 100. Так как от запуска к запуску процессорное время, затрачиваемое на выполнение алгоритма, менялось в определенном промежутке, необходимо было усреднить вычисляемые значения. Для этого каждый алгоритм запускался по K раз, и для полученных K значений определялось среднее арифметическое , которое заносилось в таблицу результатов.

Так как трудоемкость алгоритмов сортировки зависит от состояния входной последовательности, эксперимент проводился на трех видах последовательностей: отсортированных, отсортированных в обратном порядке и случайных.

Результаты эксперимента были представлены в виде таблиц и графиков, приведенных в следующем подразделе.

\section{Результаты эксперимента}

\section{Вывод}

По результатам эксперимента можно сделать следующие выводы:
\begin{itemize}[left=\parindent]
        \item
\end{itemize}

