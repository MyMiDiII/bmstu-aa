\chapter{Исследовательская часть}

\section{Технические характеристики}

Технические характеристики устройства, на котором выполнялось тестирование:

\begin{itemize}
	\item Операционная система: Manjaro \cite{manjaro} Linux x86\_64.
	\item Память: 8 GiB.
	\item Процессор: Intel® Core™ i5-8265U\cite{intel}.
\end{itemize}

Тестирование проводилось на ноутбуке, включенном в сеть электропитания. Во
время тестирования ноутбук был нагружен только встроенными приложениями
окружения, окружением, а также непосредственно системой тестирования.

\section{Примеры работы программы}

На рисунке \ref{img:exp} представлены результаты работы программы на случайной
последовательности.

\img{60mm}{img01}{Пример работы программы}{exp}

\section{Результаты тестирования}

Программа была протестирования на входных данных, приведенных в таблице
\ref{tab:tests}. Полученные результаты работы программы совпали с ожидаемыми
результатами.

\section[Постановка эксперимента по замеру времени]
        {Постановка ~~эксперимента ~~по ~~замеру времени}

Для оценки времени работы реализаций алгоритмов сортировки был проведен
эксперимент, в котором определялось влияние размеров последовательности на
время работы каждого из алгоритмов. Тестирование проводилось на размерах
последовательностей от 100 до 1000 с шагом 100. Так как от запуска к запуску
процессорное время, затрачиваемое на выполнение алгоритма, менялось в
определенном промежутке, необходимо было усреднить вычисляемые значения. Для
этого каждый алгоритм запускался по 500 раз, и для полученных 500 значений
определялось среднее арифметическое , которое заносилось в таблицу результатов.

Так как трудоемкость алгоритмов сортировки зависит от состояния входной
последовательности, эксперимент проводился на трех видах последовательностей:
отсортированных, отсортированных в обратном порядке и случайных.

Результаты эксперимента были представлены в виде таблиц и графиков, приведенных
в следующем подразделе.

\section{Результаты эксперимента}

В таблицах \ref{tab:sorted}, \ref{tab:resorted}, \ref{tab:random} представлены
результаты измерения времени работы алгоритмов на отсортированных,
отсортированных в обратном порядке и случайных последовательностях
соответственно.

На основе табличных данных построены графики зависимости времени работы каждого
алгоритма от размеров последовательностей. На рисунке \ref{img:sortall}
представлены графики зависимостей времени работы алгоритмов от длины
отсортированных последовательностей. Для наглядности графики зависимостей для
сортировки вставками и перемешиванием вынесены в отдельный рисунок
\ref{img:sort}. На рисунках \ref{img:resort} и \ref{img:random} представлены
графики зависимостей времени работы алгоритмов от длины отсортированных в
обратном порядке и случайных последовательностей соответственно.

\csvtable{sorted}{Время работы алгоритмов на отсортированных
последовательностях}{sorted}

\img{60mm}{img02}{Графики зависимости времени работы алгоритмов от размера
отсортированных последовательностей}{sortall}

\img{60mm}{img03}{Графики зависимости времени работы алгоритмов сортировки
вставками и перемешиванием от размера отсортированных
последовательностей}{sort}

\csvtable{resorted}{Время работы алгоритмов на отсортированных в обратном
порядке последовательностях}{resorted}

\img{60mm}{img04}{Графики зависимости времени работы алгоритмов от размера
отсортированных в обратном порядке последовательностей}{resort}

\csvtable{random}{Время работы алгоритмов на случайных
последовательностях}{random}

\img{60mm}{img05}{Графики зависимости времени работы алгоритмов от размера
случайных последовательностей}{random}

\clearpage
\section{Вывод}

По результатам эксперимента можно сделать следующие выводы:
\begin{itemize}[left=\parindent]
    \item на отсортированных последовательностях алгоритм сортировки выбором
        работает в 150-200 раза дольше двух других алгоритмов, так как имеет в
        лучшем случае сложность $O(N^2)$, в то время как два других алгоритма
        имеют в лучшем случае сложность $O(N)$;
    \item на отсортированных последовательностях алгоритм сортировки
        перемешиванием работает в 1.8-2 раза быстрее алгоритма сортировки
        вставками, так как при отсутствии обмена элементов сразу же
        завершается, что также отражено в трудоемкости алгоритмов в лучшем
        случае (мультипликативноая константа при $N$ у трудоемкости алгоритма
        вставками в лучшем случае примерно в 2 раза больше, чем у трудоемкости
        алгоритма перемешиванием);
    \item в худшем случае, то есть при отсортированных в обратном порядке
        последовательностях, все алгоритмы имеют квадратичную сложность, что
        отражено на соответствующем графике (рисунок \ref{img:resort}), однако
        трудоемкости имеют различные мультипликативные константы, что
        определяет их различное время работы, так алгоритм сортировки
        перемешиванием работает примерно в 1.7 раза дольше алгоритма сортировки
        вставками, и примерно в 2.7 раза дольше алгоритма сортировки выбором,
        что соответствует отношениям их мультипликативных констант при $N^2$;
    \item в случае случайных последовательностей алгоритмы сортировки вставками
        и выбором работают приблизительно одинаковое время, в то время как
        алгоритм сортировки перемешиваем работает приблизительно в 1.8 раза
        дольше.
\end{itemize}

Таким образом, для сортировки последовательностей, которые в своем большинстве
явлются неупорядоченными, из данных трех алгоритмов, стоит предпочесть алгоритм
сортировки вставками, так как на случайных последовательностях работает время
соизмеримое с алгоритмом сортировки выбором и меньшее по сравнению с алгоритмом
сортировки перемешиваем и в то же время на упорядоченных в том или ином порядке
последовательностях в среднем дает лучший результат по сравнению с алгоритмом
сортировки выбором.
