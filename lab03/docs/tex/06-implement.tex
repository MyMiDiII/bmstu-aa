\chapter{Технологическая часть}

В данном разделе описаны требования к программному обеспечению, средства
реализации, приведены листинги кода и данные, на которых будет проводиться
тестирование.

\section{Требования к программному обеспечению}

Программа должна предоставлять следующие возможности:
\begin{itemize}[left=\parindent]
    \item выбор режима работы: для единичного эксперимента и для массовых
          эксперименов;
    \item в режиме единичного эксперимента ввод последовательности для
          сортировки и вывод результатов работы каждого из алгритмов сортировки
          (отсортированной последовательности);
    \item в режиме массовых экспериментов измерение времени работы каждого из
          алгоритмов сортировки при различных длинах сортируемой
          последовательности и различных её первоначальных состояниях:
          отсортированная, отсортированная в обратном порядке, случайная
          последовательности.
\end{itemize}

\section{Средства реализации}

Для реализации данной лабораторной работы выбран интерпретируемый язык
программирования высокого уровня Python\cite{python}, так как он позволяет
реализовывать сложные задачи за кратчайшие сроки за счет простоты синтаксиса и
наличия большого количества подключаемых библиотек. 

В качестве среды разработки выбран текстовый редактор Vim\cite{vim} c
установленными плагинами автодополнения и поиска ошибок в процессе написания,
так как он реализует быстрое перемещение по тексту программы и простое
взаимодействие с командной строкой.

Замеры времени проводились при помощи функции process\_time\_ns из библиотеки
time\cite{time}.

\section{Листинги кода}

В данном подразделе представлены листинги кода ранее описанных алгоритмов:
\begin{itemize}[left=\parindent]
    \item алгоритм сортировки вставками (листинг \ref{lst:insert});
    \item алгоритм сортировки перемешиванием (листинг \ref{lst:shaker});
    \item алгоритм сортировки выбором (листинги \ref{lst:select}).
\end{itemize}

\mylisting{Реализация алгоритма сортировки вставками}{insert}{1-12}{sorts.py}
\mylisting{Реализация алгоритма сортировки
           перемешиванием}{shaker}{15-35}{sorts.py}
\mylisting{Реализация алгоритма сортировки выбором}{select}{38-48}{sorts.py}

\section{Описание тестирования}

В таблице \ref{tab:tests} приведены функциональные тесты для алгоритмов
сортировки.

\begin{table}[h!]
	\begin{center}
    \begin{threeparttable}
        \captionsetup{justification=raggedright,singlelinecheck=off}
        \caption{\label{tab:tests}Функциональные тесты}
        \begin{tabular}{|c|c|}
			\hline
            \textbf{Последовательность} & \textbf{Ожидаемый результат} \\ [2mm]
            \hline
            $[~]$
            &
            $[~]$
            \\
            \hline
            $[1, 2, 3, 4]$
            &
            $[1, 2, 3, 4]$
            \\
            \hline
            $[-2, -1, 0, 3, 4]$
            &
            $[-2, -1, 0, 3, 4]$
            \\
            \hline
            $[100, 50, -20, -99]$
            &
            $[-99, -20, 50, 100]$
            \\
            \hline
            $[13, 7, 1, -1, -17]$
            &
            $[-17, -1, 1, 7, 13]$
            \\
            \hline
            $[4, 2, -44, 76, 0, 11]$
            &
            $[-44, 0, 2, 4, 11, 76]$
            \\
            \hline
            $[113]$
            &
            $[133]$
            \\
            \hline
		\end{tabular}
    \end{threeparttable} 
	\end{center}
\end{table}

\section{Вывод}

В данном разделе были реализованы алгоритмы сортировки вставками, перемешиванием, выбором. Также были написаны тесты для каждого класса эквивлентнсти, описанного в конструкторском разделе.
