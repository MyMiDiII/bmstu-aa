\chapter*{Заключение}
\addcontentsline{toc}{chapter}{Заключение}

В ходе исследования был проведен сравнительный анализ алгоритмов, в результате
которого было выяснено, что на отсортированных последовательностях алгоритм
сортировки выбором, имеющий в данном случае сложность $O(N^2)$ ведет себя
неэффективно по сравнению с двумя другими исследуемыми алгоритмами со
сложностями $O(N)$. На последовательностях упорядоченных в обратном порядке все
алгоритмы имеют квадратичную сложность, однако наиболее эффективно себя ведет
алгоритм сортировки выбором. На случайных же последовательностях алгоритмы
сортировки выбором и вставками имеют приблизительное равное время выполения, в
то время как алгоритм сортировки перемешиванием явно им уступает.

В ходе выполения лабораторной работы:

\begin{itemize}[left=\parindent]
    \item были описаны и разработаны алгоритмы сортировки вставками,
        перемешиванием и выбором;
    \item была произведена оценка трудоемкости каждого из алгоритмов;
    \item был реализован каждый из описанных алгоритмов;
    \item по экспериментальным данным были сделаны выводы об эффективности по
        времени каждого из реализованных алгоритмов, которые были подтверждены
        теоретическими расчетами трудоемкости алгоритмов.
\end{itemize}

Таким образом, все поставленные задачи были выполнены, а цель достигнута.
