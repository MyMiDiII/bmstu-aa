\chapter{Конструкторская часть}


\section{Разработка алгоритмов}


\subsection{}


\section{Оценка трудоемкости алгоритмов}

В данном подразделе производится оценка трудоемкости каждого из алгоритмов.

\subsection{Модель вычислений}

Введем модель вычислений для оценки трудоемкости алгоритмов:
\begin{itemize}[left=\parindent]
    \item операции из списка \ref{op2} имеют трудоемкость 2;
        \begin{equation}\label{op2}
            *,~/,~//,~\%,~*=,~/=,~//=
        \end{equation}

    \item операции из списка \ref{op1} имеют трудоемкость 1;
        \begin{equation}\label{op1}
            =,~+,~-,~+=,~-=,~<,~>,~==,~!=,~>=,~<=,~[],~<<,~>>,~++,~--
        \end{equation}

    \item трyдоемкость оператора выбора \texttt{if} условие \texttt{then} А
        \texttt{else} B рассчитывается по формуле \ref{ifeq}:
        \begin{equation}\label{ifeq}
            f_{if} = f_{условия} +
            \begin{cases}
                f_A, & \text{если условие выполняется;}\\
                f_B, & \text{иначе}
            \end{cases}
        \end{equation}

    \item трудоемкость оператора цикла рассчитывется по формуле \ref{foreq}:
        \begin{equation}\label{foreq}
            f_{\text{for}} = f_{\text{инициализации}} + f_{\text{сравнения}} +
                      N(f_{\text{тела}} + f_{\text{инкремента}} +
                      f_{\text{сравнения}})
        \end{equation}

    \item трyдоемкость вызова функции равна 0.
\end{itemize}

\subsection{}

\subsection{}

\subsection{}

\section{Структура разрабатываемого ПО}

Для реализации разрабатываемого программного обеспечения будет использоваться
метод структурного программирования. Каждый из алгоритмов будет представлен
отдельной функцией, при необходимости будут выделены подпрограммы для каждой из
них. Также будут реализованы функции для ввода-вывода и функция, вызывающая все
подпрограммы для связности и полноценности программы.

\section{Классы эквивалентности при тестировании}

Для тестирования программного обеспечения во множестве тестов будут выделены
следующие классы эквивалентности:
\begin{itemize}[left=\parindent]
    \item
    \item
    \item
    \item
    \item
    \item
\end{itemize}

\section{Вывод}
