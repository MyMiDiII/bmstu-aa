\chapter{Конструкторская часть}

В данном разделе разрабатываются алгоритмы сортировки, структура программы и
способы её тестироваия, также проводится оценка трудоемкости каждого из
алгоритмов.

\section{Разработка алгоритмов}

На рисунке \ref{scheme:insertion} представлена схема алгоритма сортировки
вставками, на рисунке \ref{scheme:shaker} --- схема алгоритма сортировки
перемешиванием, на рисунке \ref{scheme:selection} --- схема алгоритма сортировки
выбором.

\noindent
\scheme{180mm}{insertionSort}{Схема алгоритма сортировки вставками}{insertion}
\noindent
\scheme{180mm}{shakerSort}{Схема алгоритма сортировки перемешиванием}{shaker}
\noindent
\scheme{180mm}{selectionSort}{Схема алгоритма сортировки выбором}{selection}

\clearpage
\section{Оценка трудоемкости алгоритмов}

В данном подразделе производится оценка трудоемкости каждого из алгоритмов.

\subsection{Модель вычислений}

Введем модель вычислений для оценки трудоемкости алгоритмов:
\begin{itemize}[left=\parindent]
    \item операции из списка \ref{op2} имеют трудоемкость 2;
        \begin{equation}\label{op2}
            *,~/,~//,~\%,~*=,~/=,~//=
        \end{equation}

    \item операции из списка \ref{op1} имеют трудоемкость 1;
        \begin{equation}\label{op1}
            \begin{aligned}
                =,~+,~-,~+=,~-=,~<,~>,~==,~!=,\\
                ~>=, ~<=,~[],~<<,~>>,~++,~--,and,or
            \end{aligned}
        \end{equation}

    \item трyдоемкость оператора выбора \texttt{if} условие \texttt{then} А
        \texttt{else} B рассчитывается по формуле \ref{ifeq}:
        \begin{equation}\label{ifeq}
            f_{if} = f_{условия} +
            \begin{cases}
                f_A, & \text{если условие выполняется;}\\
                f_B, & \text{иначе}
            \end{cases}
        \end{equation}

    \item трудоемкость оператора цикла рассчитывется по формуле \ref{foreq}:
        \begin{equation}\label{foreq}
            f_{\text{for}} = f_{\text{инициализации}} + f_{\text{сравнения}} +
                      N(f_{\text{тела}} + f_{\text{инкремента}} +
                      f_{\text{сравнения}})
        \end{equation}

    \item трyдоемкость вызова функции равна 0.
\end{itemize}

\subsection{Оценка трудоемкости алгоритма сортировки вставками}

% !!! type here

\subsection{Оценка трудоемкости алгоритма сортировки перемешиванием}

% !!! type here

\subsection{Оценка трудоемкости алгоритма сортировки выбором}

% !!! type here

\section{Структура разрабатываемого ПО}

Для реализации разрабатываемого программного обеспечения будет использоваться
метод структурного программирования. Каждый из алгоритмов будет представлен
отдельной функцией, при необходимости будут выделены подпрограммы для каждой из
них. Также будут реализованы функции для ввода-вывода и функция, вызывающая все
подпрограммы для связности и полноценности программы.

\section{Классы эквивалентности при тестировании}

Для тестирования программного обеспечения во множестве тестов будут выделены
следующие классы эквивалентности:
\begin{itemize}[left=\parindent]
    \item пустой массив; 
    \item упорядоченный массив четной длины;
    \item упорядоченный массив нечетной длины;
    \item упорядоченный в обратном порядке массив четной длины;
    \item упорядоченный в обратном порядке массив нечетной длины;
    \item случайный массив;
    \item массив из одного элемента.
\end{itemize}

\section{Вывод}

В данном разделе были разработаны алгоритмы сортировки вставками,
перемешиванием и выбором, также была произведен оценка трудоемкостей
алгоритмов. Для дальнейшей проверки правильности работы программы были выделены
классы эквивалентности тестов.
