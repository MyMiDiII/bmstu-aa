\chapter{Конструкторская часть}

В данном разделе разрабатываются алгоритмы каждого потока конвейерной
обрабокти, включая главный, также описывается структура программы и способы её
тестироваия.

\section{Разработка алгоритмов}

На рисунке \ref{scheme:main} представлена схема главного потока,
запускающего потоки этапов конвейера, схемы алгоритмов которых представлены на
рисунках \ref{scheme:first}-\ref{scheme:third}.

\scheme{120mm}{main}{Схема главного потока конвейерной обработки заявок}{main}
\noindent
\scheme{90mm}{first}{Схема потока первого этапа обработки}{first}
\noindent
\scheme{90mm}{second}{Схема потока второго этапа обработки}{second}
\noindent
\scheme{90mm}{third}{Схема потока третьего этапа обработки}{third}

\section{Структура разрабатываемого ПО}

Для реализации разрабатываемого программного обеспечения будет использоваться
метод структурного программирования. Каждый из алгоритмов будет представлен
отдельной функцией, при необходимости будут выделены подпрограммы для каждой из
них. Также будут реализованы функции для ввода-вывода и функция, вызывающая все
подпрограммы для связности и полноценности программы.

\section{Классы эквивалентности при тестировании}

Для тестирования программного обеспечения во множестве тестов будут выделены
следующие классы эквивалентности:
\begin{itemize}[left=\parindent]
    \item отрицательное число;
    \item ноль заявок; 
    \item нечисловые данные;
    \item отрицательное время;
    \item нулевое время;
    \item нулевой разброс;
    \item простой тест.
\end{itemize}

\section{Вывод}

В данном разделе были разработаны алгоритмы конвейерной обработки заявок, была
описана структура разрабатываемого ПО. Для дальнейшей проверки правильности
работы программы были выделены классы эквивалентности тестов.
