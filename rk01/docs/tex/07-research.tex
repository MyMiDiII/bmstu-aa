\chapter{Исследовательская часть}

\section{Технические характеристики}

Технические характеристики устройства, на котором выполнялось тестирование:

\begin{itemize}
	\item Операционная система: Manjaro \cite{manjaro} Linux x86\_64.
	\item Память: 8 GiB.
    \item Процессор: Intel® Core™ i5-8265U, 4 физических ядра, 8 логических
        ядра\cite{intel}.
\end{itemize}

Тестирование проводилось на ноутбуке, включенном в сеть электропитания. Во
время тестирования ноутбук был нагружен только встроенными приложениями
окружения, окружением, а также непосредственно системой тестирования.

\section{Примеры работы программы}

На рисунке \ref{img:log} представлены результаты работы программы.

\img{220mm}{log}{Пример работы программы}{log}


\section{Результаты тестирования}

Программа была протестирована на входных данных, приведенных в таблице
\ref{tab:tests}. Полученные результаты работы программы совпали с ожидаемыми
результатами.

\section[Постановка эксперимента по замеру времени]
        {Постановка ~~эксперимента ~~по ~~замеру времени}

Для оценки зависимости времени простоя каждой ленты от разброса заданных
значений времени и соотношениями между срeдними значениями времени обработки
заявок был проведен эксперимент. Так как каналы для передачи заявок между
потоками не позволяют получить время обслуживающих действий, для получения
достоверных результатов необходимо такое время обработки заявок, для которого
время обслуживающие действия мало. Для этого программе подаются различные
порядки среднего времени обработки заявки (одинаковые для всех трех лент без
разброса для исключения ситуации простоя), находится разница между ожидаемым
временем обработки всех заявок и полученным, если эта разница мала по сравнению
со значением ожидаемого времени, то заданный порядок можно использовать
для анализа (эксперимент 1).

Эксперимент по оценке зависимости времени простоя лент от разброса был проведен
на одинаковых значениях среднего времени обрабоки на всех лентах и разбросах от
10 до 810 с шагом 100 (в миллисекундах) (эксперимент 2).

Для анализа зависимости времени простоя от соотношений средних значений времени
обработки программа была протестирована на следующих соотноешниях: 1:1:1,
2:1:1, 1:2:1, 1:1:2, 2:2:1, 1:2:2, 2:1:2 (эксперимент 3).

Результаты эксперимента были представлены в виде таблиц и графиков, приведенных
в следующем подразделе.

\section{Результаты эксперимента}

На листинге \ref{lst:time} представлены результаты провдения эксперимента 1.

На листинге \ref{lst:delta} представлены результаты провдения эксперимента 2.
На рисунке \ref{img:delta} представлены графики зависимости простоя 2-ой и 3-ей
лент (первая лента не представлена потому, что она не простаивает, так как все
заявки сразу передаются в её входную очередь).

\img{90mm}{delta}{Графики зависимости времени простоя второй и третьей лент}{delta}

На листинге \ref{lst:rations} представлены результаты провдения эксперимента 3.

\mylisting{Результаты эксперимента по определению порядка среднего
времени}{time}{2-15}{graphics/log.txt}

\begin{mdlisting}
    \captionsetup{justification=raggedright,singlelinecheck=off}
    \lstinputlisting[label=lst:delta, caption=Результаты эксперимента по
    определению зависимости времени простоя от разброса средних значений
    времени,
    linerange=2-37]{../../src/graphics/log.txt}
\end{mdlisting}

\begin{mdlisting}
    \captionsetup{justification=raggedright,singlelinecheck=off}
    \lstinputlisting[label=lst:rations, caption=Результаты эксперимента по
    определению зависимости времени простоя каждой ленты от соотношения средних
    значений времени,
    linerange=58-92]{../../src/graphics/log.txt}
\end{mdlisting}

\section{Вывод}

По результатам эксперимента можно сделать следующие выводы:
\begin{itemize}[left=\parindent]
    \item на данной машине время обслуживающие действия становятся мало при
        времени обработки заявок от 1 секуды;
    \item при увеличении разброса значений времени обработки заявок
        увеличивается время простоя второй и третьей лент;
    \item первая лента не простаивает, так как все заявки подаются на её
        входную очередь одновременно;
    \item при соотношении средних значений времени обработки заявки 1:1:1
        вторая и третья лента простаивают малое время за счет обрабатывающих
        действий на первой ленте;
    \item при соотношении средних значений времени обработки заявки 2:1:1
        вторая и третья лента простаивают значительное время, так как
        после обработки поданной заяки ожидают, пока свою обработку
        завершит первая лента;
    \item при соотношении средних значений времени обработки заявки 1:2:1
        простаивает третья лента, так как ждет пока заявку обработает вторая
        лента;
    \item при соотношении средних значений времени обработки заявки 1:1:2 ленты
        не простаивают, так как третья лента передает свои заявки на выход
        конвейера и завершения ею обрабоки заявки другие ленты не ждут;
    \item при соотношении средних значений времени обработки заявки 2:2:1
        простаивает третья лента, так как ждет завершения обработки заявок
        первыми двумя лентами;
    \item при соотношении средних значений времени обработки заявки 1:2:2
        ленты не простаивают;
    \item при соотношении средних значений времени обработки заявки 2:1:2
        простаивает 2 лента, так ждет завершения обработки заявок первой
        лентой.
\end{itemize}

