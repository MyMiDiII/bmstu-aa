\chapter{Аналитическая часть}

В данном разделе представлено теоретическое описание моделирования
конвейерной обработки данных.

\section{Конвейерная обработка данных}

Модель конвейерной обработки данных была описана в лабораторной работе №5
\cite{myReport}, в данной работе, так как на вход программе подаются значения
времени обработки заявок на каждой из лент, на этапах не будет выполняться
никакой полезной работы, на каждом потоке будет смоделирована обработка с
помощью задержки на заранее сгенерированное время обработки заявки, попадающее
в заданный диапазон.

\section{Определение времени простоя лент}

Время работы каждой ленты конвейера затрачивается на одну из следующих
операций:
\begin{itemize}
    \item обработка заявки;
    \item обслуживающие действия (получение заявки из входной очереди, отправка
          заявки на следующий этап);
    \item простой (ожидание следующей заявки).
\end{itemize}

Будем считать, что лента начинает работать с момента приема первой заявки
$t_{\text{нач}}^{i}$ и до момента окончания обработки последней заявки
$t_{\text{кон}}^{i}$. 

Таким образом, время простоя каждой ленты можно вычислить по формуле
\ref{eq:11}:
\begin{equation}\label{eq:11}
    t_{\text{простоя}} = t_{\text{кон}}^{i} - t_{\text{нач}}^{i}
                         - t_{\text{обработки}} - t_{\text{обслуживания}}
\end{equation}

\section{Вывод}

В данном разделе была кратко описана модель реализуемой конвейерной обработки,
также было представлено математическое описание поиска времени простоя каждой
из лент. Из представленных описаний можно предъявить следующие требования к
разрабатываемому программному обеспечению: по заданным количеству заявок и
времени обработки заявки на каждой ленте программа должна выдавать лог работы
каждой ленты, а также время их простоя, при неверных входных данных программа
должна выдавать сообщение об ошибке.
