\chapter{Конструкторская часть}

В данном разделе разрабатываются алгоритмы, а также производится сравнительный
анализ рекурсивной и нерекурсивной их реализации на примере алгоритма поиска
расстояния Левенштейна.

\section{Разработка алгоритмов}

В данном подразделе приводятся схемы разработанных алгоритмов поиска расстояния
Левенштейна и Дамерау-Левенштейна.

\subsection[Схемы алгоритмов поиска расстояния Левенштейна]
           {Схемы ~~~~алгоритмов ~~~~поиска ~~~~расстояния Левенштейна}

Схема рекурсивного алгоритма поиска расстояния Левенштейна представлена
на рисунке \ref{scheme:recLiv}.
\noindent
\scheme{170mm}{recursiveLevenstein}{Схема рекурсивного алгоритма
            поиска расстояния Левенштейна}{recLiv}

В двух следующих алгоритмах используется матрица для сохранения результатов
вычислений, так как в обоих матрица инициализируется одним и тем же способом
на рисунке \ref{scheme:matrInit} представен алгоритм инициализации матрицы.
\noindent
\scheme{135mm}{matrixInit}{Схема алгоритма инициализации матрицы}{matrInit}

Матричный алгоритм и рекурсивный алгоритм с кэшем для поиска расстояния
Левенштейна приведены на рисунках \ref{scheme:matrLev} и
\ref{scheme:cacheLev}-\ref{scheme:recCacheLev} соответственно.

\noindent
\scheme{183mm}{matrixLevenstein}{Схема матричного алгоритма поиска расстояния
Левенштейна}{matrLev}
\noindent
\scheme{55mm}{cacheLevenstein}{Схема рекурсивного алгоритма поиска расстояния
Левенштейна с кэшем}{cacheLev}
\noindent
\scheme{172mm}{recursiveCacheLevenstein}{Схема рекурсивной подпрограммы
алгоритма поиска расстояния Левенштейна с кэшем}{recCacheLev}

\subsection[Схема алгоритма поиска расстояния Дамерау-Левенштейна]
           {Схема ~~~~алгоритма ~~~~поиска ~~~~расстояния Дамерау-Левенштейна}

Схема рекурсивного алгоритма поиска расстояния Дамерау-Левенштейна представлена
на рисунке \ref{scheme:recDamLiv}.
\noindent
\scheme{210mm}{recursiveDamerauLevenstein}{Схема рекурсивного алгоритма
               поиска расстояния Дамерау-Левенштейна}{recDamLiv}
