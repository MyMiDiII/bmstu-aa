\chapter{Исследовательская часть}

\section{Технические характеристики}

Технические характеристики устройства, на котором выполнялось тестирование:

\begin{itemize}
	\item Операционная система: Manjaro \cite{manjaro} Linux \cite{linux} x86\_64.
	\item Память: 8 GiB.
	\item Процессор: Intel® Core™ i5-8265U\cite{intel}.
\end{itemize}

Тестирование проводилось на ноутбуке, включенном в сеть электропитания. Во время тестирования ноутбук был нагружен только встроенными приложениями окружения, окружением, а также непосредственно системой тестирования.

\section{Результаты тестирования}

В таблице \ref{tab:testRes} приведены результаты работы программы на тестах,
описанных в таблице \ref{tab:tests}. В результате сравнения ожидаемого и
полученного результата делаем вывод, что все тесты были пройдены.

\begin{table}[h]
	\begin{center}
		\caption{\label{tab:testRes}Результаты тестирования}
		\begin{tabular}{|c|c|c|c|}
			\hline
			& & \multicolumn{2}{c|}{\bfseries Полученный результат}\\ \cline{3-4}
			\bfseries Строка 1  & \bfseries Строка 2 &
            \bfseries Левенштейн & \bfseries Дамерау-Левенштейн
			\csvreader{../data/csv/tests.csv}{}
			{\\\hline \csvcoli&\csvcolii&\csvcoliii&\csvcoliv}
			\\\hline
		\end{tabular}
	\end{center}
\end{table}
