\chapter{Технологическая часть}

В данном разделе описаны требования к программному обеспечению, средства реализации, приведены листинги кода и данные, на которых будет проводиться тестирование.

\section{Требования к программному обеспечению}

Программа должна предоставлять следующие возможности:
\begin{itemize}[left=\parindent]
    \item выбор режима работы: для единичного эксперимента и для массовых
          эксперименов;
    \item в режиме единичного эксперимента ввод двух строк на русском или
          английском языках и вывод полученных разными реализациями расстояний;
    \item в режиме массовых экспериментов измерение времени при различных
          длинах строк и построение графиков по полученным данным.
\end{itemize}

\section{Средства реализации}

% ! Ссылки на сайты!!!

Для реализации данной лабораторной работы выбран интерпретируемый язык
программирования высокого уровня Python, так как он позволяет реализовывать
сложные задачи за кратчайшие сроки за счет простоты синтаксиса и наличия
большого количества подключаемых библиотек. 

В качестве среды разработки выбран текстовый редактор vim c установленными
плагинами автодополнения и поиска ошибок в процессе написания, так как он
реализует быстрое перемещение по тексту программы и простое взаимодействие с
командной строкой.

\section{Листинги кода}

\section{Описание тестирования}
