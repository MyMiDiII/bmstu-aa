\chapter{Технологическая часть}

В данном разделе описаны требования к программному обеспечению, средства
реализации, приведены листинги кода и данные, на которых будет проводиться
тестирование.

\section{Требования к программному обеспечению}

Программа должна предоставлять следующие возможности:
\begin{itemize}[left=\parindent]
    \item выбор режима работы: для единичного эксперимента и для массовых
          эксперименов;
    \item в режиме единичного эксперимента ввод двух строк на русском или
          английском языках и вывод полученных разными реализациями расстояний;
    \item в режиме массовых экспериментов измерение времени при различных
          длинах строк и построение графиков по полученным данным.
\end{itemize}

\section{Средства реализации}

% ! Ссылки на сайты!!!

Для реализации данной лабораторной работы выбран интерпретируемый язык
программирования высокого уровня Python, так как он позволяет реализовывать
сложные задачи за кратчайшие сроки за счет простоты синтаксиса и наличия
большого количества подключаемых библиотек. 

В качестве среды разработки выбран текстовый редактор vim c установленными
плагинами автодополнения и поиска ошибок в процессе написания, так как он
реализует быстрое перемещение по тексту программы и простое взаимодействие с
командной строкой.

\section{Листинги кода}

В данном подразделе представлены листинги кода ранее описанных алгоритмов:
\begin{itemize}[left=\parindent]
    \item рекурсивный алгоритм поиска расстояния Левенштейна (листинг
          \ref{lst:recLev});
    \item матричный алгоритм поиска расстояния Левенштейна (листинги
          \ref{lst:matInit}-\ref{lst:matLev});
    \item рекурсивный алгоритм поиска расстояния Левенштейна с кэшем (листинги
          \ref{lst:matInit}, \ref{lst:recCacheLevInit}-\ref{lst:recCacheLev});
    \item рекурсивный алгоритм поиска расстояния Дамерау-Левенштейна (листинг
          \ref{lst:recDamLev}).
\end{itemize}

\begin{lstinputlisting}[
	caption={Реализация рекурсивного алгоритма поиска расстояния Левенштейна},
	label={lst:recLev},
	linerange={20-35}
]{../../src/levenstein.py}
\end{lstinputlisting}

\begin{lstinputlisting}[
	caption={Реализация инициализации матрицы},
	label={lst:matInit},
	linerange={3-7}
]{../../src/levenstein.py}
\end{lstinputlisting}

\begin{lstinputlisting}[
	caption={Реализация матричного алгоритма поиска расстояния Левенштейна},
	label={lst:matLev},
	linerange={38-52}
]{../../src/levenstein.py}
\end{lstinputlisting}

\begin{lstinputlisting}[
    caption={Реализация инициализации данных для рекурсивного алгоритма поиска
             расстояния Левенштейна с кэшем},
	label={lst:recCacheLevInit},
	linerange={77-82}
]{../../src/levenstein.py}
\end{lstinputlisting}

\begin{lstinputlisting}[
	caption={Реализация рекурсивного алгоритма поиска расстояния Левенштейна с кэшем},
	label={lst:recCacheLev},
	linerange={55-74}
]{../../src/levenstein.py}
\end{lstinputlisting}

\begin{lstinputlisting}[
    caption={Реализация рекурсивного алгоритма поиска расстояния
             Дамерау-Левенштейна с кэшем},
	label={lst:recDamLev},
	linerange={3-27}
]{../../src/dameraulevenstein.py}
\end{lstinputlisting}

\section{Описание тестирования}

В таблице \ref{tab:tests} приведены функциональные тесты для алгоритмов поиска
расстояний Левенштейна и Дамерау-Левенштейна.

\begin{table}[h]
	\begin{center}
		\caption{\label{tab:tests}Функциональные тесты}
		\begin{tabular}{|c|c|c|c|}
			\hline
			& & \multicolumn{2}{c|}{\bfseries Ожидаемый результат}\\ \cline{3-4}
			\bfseries Строка 1  & \bfseries Строка 2 &
            \bfseries Левенштейн & \bfseries Дамерау-Левенштейн
			\csvreader{../data/csv/tests.csv}{}
			{\\\hline \csvcoli&\csvcolii&\csvcoliii&\csvcoliv}
			\\\hline
		\end{tabular}
	\end{center}
\end{table}
