\chapter*{Введение}
\addcontentsline{toc}{chapter}{Введение}

\textit{Расстояние Левенштейна} --- минимальное количество операций вставки,
удаления и замены символа, необходимых для превращения одной строки в другую.
Если к указанным операциям добавить перестановку двух соседних символов,
получим определение \textit{расстояния Дамерау-Левенштейна} \cite{distances}.
Поиск каждой из этих характеристик основан на рекуррентных вычислениях, то есть
на вычислениях, которые используют уже вычисленные значения для вычисления
новых.

Таким образом, задача поиска данных расстояний основывается на методе
динамического программирования --- разбиении задач на более мелкие и простые
подзадачи такого же вида, решение которых проводится один раз и далее
используется при решении других задач и подзадач \cite{dynamic}. Поэтому
изучение, разработка и реализация алгоритмов поиска расстояний Левенштейна и
Дамерау"=Левенштейна позволит получить навыки использования данного метода.

\textbf{Целью данной работы} является изучение метода динамического
программирования на материале алгоритмов Левенштейна и Дамерау"=Левенштейна. 

Для достижения поставленной цели необходимо выполнить следующие
\textbf{задачи}:
\begin{itemize}[left=\parindent]
    \item изучить алгоритмы Левенштейна и Дамерау-Левенштейна нахождения
          расстояния между строками;
    \item разработать алгоритмы поиска расстояния между строками;
    \item оценить объем используемой алгоритмами памяти;
    \item определить средства программной реализации;
    \item реализовать указанные алгоритмы;
    \item провести тестирование реализованных алгоритмов;
    \item провести сравнительный анализ алгоритмов и процессорному времени
          работы реализации.
\end{itemize}
