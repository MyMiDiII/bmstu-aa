\chapter{Аналитическая часть}

В данном разделе представлено теоретическое описание алгоритмов поиска
расстояния Левенштейна и Дамерау"=Левенштейна, а также рассмотрены области их
применения.

\section{Расстояние Левенштейна}

Расстояние Левенштейна\cite{bib01} между двумя строками --- это минимальное
количество операций вставки, удаления и замены символа, необходимых для
превращения одной строки в другую.

Цена операции может зависеть от её вида и/или от участвующх в ней символов, что
отражает разную вероятность различных ошибок при вводе текста и т.~п. Для
решения задачи поиска расстояния между двумя страками необходимо найти
последовательность применяющихся операций, такую, что суммарная их цена будет
минимальной. При вычислении расстояния Левенштейна используются следующие цены:

\begin{itemize}
    \item $w(a, a) = 0$ --- цена совпадения двух символов;
    \item $w(a, b) = 1,~a \neq b$ --- цена замены символа a на символ b;
    \item $w(\lambda, a) = 1$ --- цена вставки символа a;
    \item $w(a, \lambda) = 1$ --- цена удаления символа a.
\end{itemize}

\subsection{Рекурсивный алгоритм}

В рекурсивном алгоритме поиска расстояния Левенштейна между двумя строками
искомая величина вычисляется через соответсвующие величины подстрок, а
рекуррентная формула выводится из следующих рассуждений:
\begin{enumerate}[label=\arabic*)]
    \item для перевода пустой строки в пустую требуется ноль операций; 
    \item для перевода пустой строки в строку $s$ требуется $|s|$ операций
вставки (здесь и далее $|s|$ обозначает длину строки);
    \item для перевода строки $s$ в пустую строку требуется $|s|$ операций
удаления;
    \item для перевода строки $s_1$ в строку $s_2$ требуется выполнить
некоторую последовательность операций удаления, вставки или замены, при этом
операции в оптимальной последовательности можно произвольно менять местами, так
как две последовательные операции любых видов можно переставить, что
доказывается простым перебором вариантов возможных пар, поэтому без ограничения
общности можно считать, что операция над последним символом была произведена
последней и цена преобразования строки $s_1$ в строку $s_2$ будет являться
минимальной ценой из цен, полученных одним из следующих способов (пусть
при этом $s_1'$ и $s_2'$ --- строки $s_1$ и $s_2$ без последнего символа,
соответственно):
        \begin{itemize}[itemindent=\parindent,leftmargin=\parindent]
            \item сумма цены преобразования строки $s_1$ в $s_2'$  и цены
        проведения операции вставки, которая необходима для преобразования $s_2'$
        в $s_2$;
            \item сумма цены преобразования строки $s_1'$ в $s_2$ и цены проведения
        операции удаления, которая необходима для преобразования $s_1$ в $s_1'$;
            \item сумма цены преобразования из $s_1'$ в $s_2'$ и операции замены,
        предполагая, что $s_1$ и $s_2$ оканчиваются на разные символы;
            \item цена преобразования из $s_1'$ в $s_2'$, предполагая, что $s_1$ и
        $s_2$ оканчиваются на один и тот же символ.
        \end{itemize}
\end{enumerate}


Таким образом, для расчета расстояния Левенштейна между двумя строками $s_1$ и $s_2$ используется рекуррентная формула для расчета через подстроки:
\begin{equation}
D(s_1[1..i],s_2[1..j]) =
    \begin{cases}
        0,~~i=j=0\\
        j,~~i=0,~j>0\\
        i,~~j=0,~i>0\\
        \min \{\\
         \qquad D(s_1[1..i],s_2[1..j-1]) + 1,\\
         \qquad D(s_1[1..i-1],s_2[1..j]) + 1,\\
         \qquad D(s_1[1..i-1],s_2[1..j-1]) + l(s_1[i], s_2[j])\\
        \}
    \end{cases},
\end{equation}
где величина $l(a, b)$ выражется формулой:
\begin{equation}
l(a, b) =
    \begin{cases}
        0, & \text{если}~a = b\\
        1, & \text{иначе}
    \end{cases}.
\end{equation}



\subsection{Матричный алгоритм}

\subsection{Рекурсивный алгоритм с кэшем}

\section{Расстояние Дамерау-Левенштейна}

\subsection{Рекурсивный алгоритм}

\section{Области применения алгоритмов}


