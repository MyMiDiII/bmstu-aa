\chapter{Аналитическая часть}

В данном разделе представлено теоретическое описание алгоритмов поиска
расстояния Левенштейна и Дамерау"=Левенштейна, а также рассмотрены области их
применения.

\section{Расстояние Левенштейна}

Расстояние Левенштейна\cite{bib01} между двумя строками --- это минимальное
количество операций вставки, удаления и замены символа, необходимых для
превращения одной строки в другую.

Цена операции может зависеть от её вида и/или от участвующх в ней символов, что
отражает разную вероятность различных ошибок при вводе текста и т.~п. Для
решения задачи поиска расстояния между двумя страками необходимо найти
последовательность применяющихся операций, такую, что суммарная их цена будет
минимальной. При вычислении расстояния Левенштейна используются следующие цены:

\begin{itemize}
    \item $w(a, a) = 0$ --- цена совпадения двух символов;
    \item $w(a, b) = 1,~a \neq b$ --- цена замены символа a на символ b;
    \item $w(\lambda, a) = 1$ --- цена вставки символа a;
    \item $w(a, \lambda) = 1$ --- цена удаления символа a.
\end{itemize}

\subsection{Рекурсивный алгоритм}

В рекурсивном алгоритме поиска расстояния Левенштейна между двумя строками
искомая величина вычисляется через соответсвующие величины подстрок, а
рекуррентная формула выводится из следующих рассуждений:
\begin{enumerate}[label=\arabic*)]
    \item для перевода пустой строки в пустую требуется ноль операций; 
    \item для перевода пустой строки в строку $s$ требуется $|s|$ операций
вставки (здесь и далее $|s|$ обозначает длину строки);
    \item для перевода строки $s$ в пустую строку требуется $|s|$ операций
удаления;
    \item для перевода строки $s_1$ в строку $s_2$ требуется выполнить
некоторую последовательность операций удаления, вставки или замены, при этом
операции в оптимальной последовательности можно произвольно менять местами, так
как две последовательные операции любых видов можно переставить, что
доказывается простым перебором вариантов возможных пар, поэтому без ограничения
общности можно считать, что операция над последним символом была произведена
последней и цена преобразования строки $s_1$ в строку $s_2$ будет являться
минимальной ценой из цен, полученных одним из следующих способов (пусть
при этом $s_1'$ и $s_2'$ --- строки $s_1$ и $s_2$ без последнего символа,
соответственно):
        \begin{itemize}[itemindent=\parindent,leftmargin=\parindent]
            \item сумма цены преобразования строки $s_1$ в $s_2'$  и цены
        проведения операции вставки, которая необходима для преобразования $s_2'$
        в $s_2$;
            \item сумма цены преобразования строки $s_1'$ в $s_2$ и цены проведения
        операции удаления, которая необходима для преобразования $s_1$ в $s_1'$;
            \item сумма цены преобразования из $s_1'$ в $s_2'$ и операции замены,
        предполагая, что $s_1$ и $s_2$ оканчиваются на разные символы;
            \item цена преобразования из $s_1'$ в $s_2'$, предполагая, что $s_1$ и
        $s_2$ оканчиваются на один и тот же символ.
        \end{itemize}
\end{enumerate}


Таким образом, для расчета расстояния Левенштейна между двумя строками $s_1$ и $s_2$ используется рекуррентная формула для расчета через подстроки:
\begin{equation}\label{eq11}
D(s_1[1..i],s_2[1..j]) =
    \begin{cases}
        0,~~i=j=0\\
        j,~~i=0,~j>0\\
        i,~~j=0,~i>0\\
        \min \{\\
         \qquad D(s_1[1..i],s_2[1..j-1]) + 1,\\
         \qquad D(s_1[1..i-1],s_2[1..j]) + 1,\\
         \qquad D(s_1[1..i-1],s_2[1..j-1]) + l(s_1[i], s_2[j])\\
        \}
    \end{cases},
\end{equation}
где величина $l(a, b)$ выражется формулой:
\begin{equation}\label{eq1.2}
l(a, b) =
    \begin{cases}
        0, & \text{если}~a = b\\
        1, & \text{иначе}
    \end{cases}.
\end{equation}

\subsection{Матричный алгоритм}

При явной реализации формулы \ref{eq11} через рекурсию многие вызовы будут
производиться при одних и тех же значениях параметров, то есть большое 
количество вычислений будет повторяться и не один раз. Данную проблему
решает матричный алгоритм поиска расстояния Левенштейна, который представляет
собой построчное заполнение матрицы с размерами $N+1$ на $M+1$, где
$N$ и $M$ --- размеры исходной $s_1$ и получаемой $s_2$ строк соответственно, а
в ячейке с координатами $(i,j)$ находится расстояние между подстрокой исходной
строки длины $i$, идущей от начала строки, и подстрокой получаемой строки 
длины $j$, также идущей от начала строки.

\subsection{Рекурсивный алгоритм с кэшем}

Проблему повторяющихся вычислений можно решить и при использовании
рекурсии. Для этого достаточно создать кэш в виде матрицы, где будут храниться
уже вычисленные значения. Если при выполении рекурсии происходит вызов с теми
данными, которые ещё не были обработаны, то необходимое значение вычисляется и 
заносится в соответствующую ячейку матрицы. Если же данные уже были обработаны
в дальнейших вычислениях участвует значение из матрицы. Таким образом,
повторных вычислений не происходит.

\section{Расстояние Дамерау-Левенштейна}

Расстояние Дамерау-Левенштейна\cite{bib01} между двумя строками --- это
минимальное количество операций ставки, удаления, замены и транспозиции
(перестановки двух соседних) символов, необходимых для превращения одной строки
в другую.

\subsection{Рекурсивный алгоритм}

Рекурсивный алгоритм поиска расстояния Дамерау-Левенштейна полностью аналогичен
алгоритму поиска расстояния Левенштейна, за исключением дополнительной операции
транспозиции. Для её учета в формулу \ref{eq11} добавляется рассмотрение
случаев с возможной транспозицией последних символов. После добавления
выражений, учитывающих дополнительную операции, получаем математическое
представление поиска расстояния Дамерау-Левенштейна:

{\small
\begin{equation}\label{eq13}
D(s_1[1..i],s_2[1..j]) =
    \begin{cases}
        0,~~i=j=0\\
        j,~~i=0,~j>0\\
        i,~~j=0,~i>0\\
        \min \{\\
         \qquad D(s_1[1..i],s_2[1..j - 1])+1,\\
         \qquad D(s_1[1..i-1],s_2[1..j])+1,\\
         \qquad D(s_1[1..i-1],s_2[1..j-1]) + l(s_1[i],s_2[i]), \\
         \qquad D(s_1[1..i-2],s_2[1..j-2]) + 1,\\
        \}, \text{ если } i,j > 1 \text{ и } s_1[i]=s_2[j-1] \text{ и }
            s_1[i-1]=s_2[j]\\
        \min \{\\
         \qquad D(s_1[1..i],s_2[1..j-1]) + 1,\\
         \qquad D(s_1[1..i-1],s_2[1..j]) + 1,\\
         \qquad D(s_1[1..i-1],s_2[1..j-1]) + l(s_1[i], s_2[j])\\
        \}, \text{ иначе}
    \end{cases}.
\end{equation}
}

\subsection{Рекурсивный алгоритм с кешем}

Также как и рекурсивный алгоритм полностью аналогичен соответсвующему алгоритму
поиска расстояния Левенштейна. За счет дополнительной возможной операции
в этом алгоритме для поиска расстояния от текущих подстрок анализируется 
дополнительная ячейка с индексами $i - 2$ и $j - 2$, где $i > 1$ и $j > 1$ ---
длины текущих подстрок.

\section{Области применения алгоритмов}

Поиск расстояний Левенштейна и Дамерау-Левенштейна широко используется в теории
информации и компьютерной лингвистике. Так, он применяется:
\begin{itemize}
    \item для автозамены, исправленя ошибок в словах (поисковые системы, базы
          данных, ввод текста, автоматическое распознавание отсканированного
          текса или речи);
    \item для сравнения текстовых файлов утилитой \textit{diff};
    \item в биоинформатике для сравнения генов, хромосом и белков.
\end{itemize}
