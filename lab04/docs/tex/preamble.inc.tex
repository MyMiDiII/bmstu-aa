% russian language
\usepackage[utf8]{inputenc}
\usepackage[T2A]{fontenc}
\usepackage[english,russian]{babel}

% searchable and copyable
\usepackage{cmap}

% Times New Roman
\usepackage{pscyr}
\renewcommand{\rmdefault}{ftm}

% Margins
\usepackage{geometry}
\geometry{left=30mm}
\geometry{right=10mm}
\geometry{top=20mm}
\geometry{bottom=20mm}

% Titles
\usepackage{titlesec}
\titlespacing*{\chapter}{\parindent}{-30pt}{21pt}
\titlespacing*{\section}{\parindent}{*2}{*2}
\titlespacing*{\subsection}{\parindent}{*2}{*2}
\titleformat{\chapter}{\LARGE\bfseries}{\thechapter}{18pt}{\LARGE\bfseries}
\titleformat{\section}{\Large\bfseries}{\thesection}{16pt}{\Large\bfseries}

% Spacing
\usepackage{setspace}
\usepackage{indentfirst}
\setlength{\parindent}{1.25cm}
\linespread{1.25}

% Links
\usepackage[unicode, pdftex]{hyperref}
\def\UrlBreaks{\do\/\do-}
\usepackage[nottoc]{tocbibind} % for bib link

% Listings
\usepackage{listings}
\usepackage[newfloat]{minted}
\usepackage{verbatim}

\newcommand{\mylisting}[4] {
    \noindent
    \begin{minipage}{\linewidth}
        \captionsetup{justification=raggedright,singlelinecheck=off}
        \begin{lstinputlisting}[
            caption={#1},
            label={lst:#2},
            linerange={#3}
        ]{../../src/#4}
        \end{lstinputlisting}
    \end{minipage}
}

% Images
\usepackage{graphicx}
\newcommand{\scheme}[4] {
	\begin{figure}[ht!]
		\center{
            \includegraphics[width=#1]{../data/schemes/#2}
            \caption{#3}
            \label{scheme:#4}
        }
	\end{figure}
}

\newcommand{\img}[4] {
	\begin{figure}[ht!]
		\center{
            \includegraphics[height=#1]{../data/img/#2}
            \caption{#3}
            \label{img:#4}
        }
	\end{figure}
}

% Captions
\usepackage[figurename=Рисунок,labelsep=endash,
            justification=centering]{caption}
\usepackage{hyphenat}

% Lists
\usepackage{enumitem}
\renewcommand{\labelitemi}{$-$}
\setlist{nosep, leftmargin=\parindent, wide}

% Comments
\usepackage{comment}

% Text
\usepackage{amstext}
\usepackage{ulem}
\renewcommand{\ULdepth}{1.5pt}

% Formulas
\usepackage{amsmath}

% Tabulars
\usepackage{threeparttable}
\usepackage{csvsimple}
\newcommand{\specialcell}[2][c]{%
  \begin{tabular}[#1]{@{}c@{}}#2\end{tabular}}

\newcommand{\csvtable}[3]{
    \begin{table}[h]
        \begin{center}
        \begin{threeparttable}
            \captionsetup{format=hang,justification=raggedright,
                          singlelinecheck=off}
            \caption{\label{tab:#1}#2}
            \begin{tabular}{|r|r|r|r|}
                \hline
                \bfseries Размер & \bfseries Вставками, нс &
                \bfseries Перемешиванием, нс & \bfseries Выбором, нс
                \csvreader{../data/csv/#3.csv}{}
                {\\\hline \csvcoli&\csvcolii&\csvcoliii&\csvcoliv}
                \\\hline
            \end{tabular}
        \end{threeparttable}
        \end{center}
    \end{table} 
}
