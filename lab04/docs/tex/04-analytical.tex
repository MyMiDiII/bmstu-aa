\chapter{Аналитическая часть}

В данном разделе представлено теоретическое описание алгоритмов численного
интегрирования методом средних прямоугольников.

\section{Последовательный алгоритм}

Пусть требуется вычислить определенный интеграл \ref{eq1}:

\begin{equation}\label{eq1}
    I = \int\limits_a^b f(x)dx,
\end{equation}

Суть всех методов численного интегрирования, в том числе и метода средних
прямоугольников, состоит в замене подынтегральной функции $f(x)$
вспомогательной, интеграл от которой легко вычисляется в элементарных функциях.

Предположим, что $f(x)$ непрерывна на $[a,b]$, $n$ -- натуральное и $\Delta x =
\frac{b-a}{n}$. Разделим интервал $[a,b]$ на $n$ подынтервалов длиной $\Delta
x$ и найдем среднюю точку $m_i$ каждого $i$-ого подыинтервала\cite{int}. Тогда
определенный интеграл может быть вычислен по формуле \ref{eq2}:
    
\begin{equation}\label{eq2}
    I_n = \sum\limits_{i=1}^{n} f(m_i) \Delta x,
\end{equation}

При этом, чем больше $n$, тем ближе вычисленное значение,
к реальному значению интеграла, то есть \ref{eq3}:

\begin{equation}\label{eq3}
    I = \lim\limits_{n \to \infty} I_n,
\end{equation}

Понятно, что с технической точки зрения нельзя разделить интервал на
бесконесное число подынтервалов. Это и не требуется, так как необходимой
точности вычислений можно достичь и при конечном $n$. Для этого интервал сначала
разбивают на $m$ и $m + 1$ (начиная с $m=2$) подынтервала, применяют численный
метод для каждого количества и вычисляют разницу между ними, если разница
меньше заданной точности $\varepsilon$, то вычисления прекращают, а результатом
является последнее вычисленное занчение.

\section{Параллельный алгоритм}

В алгоритме численного интегрирования методом средних прямоугольников
вычисления на каждом из подынтервалов происходят независимо, поэтому есть
возможность произвести распараллеливание данных вычислений. Количество
отрезков, на которых производит вычисление один поток, будет определяться
количеством потоков, а итоговое значение интеграла будет храниться в
разделяемой переменной, доступ к которой будут иметь все потоки, каждый из
которых будет прибавлять вычисленную им сумму к итоговому результату.

\section{Вывод}

В данном разделе был рассмотрен алгоритм численного интегрирования методом
средних прямоугольнико, так же был описан механизм распараллеливания данного
алгоритма. Из представленных описаний можно предъявить ряд требований к
разрабатываемому программному обеспечению:
\begin{itemize}[left=\parindent]
    \item на вход должны подаваться пределы интегрирования, заданная точность,
          а также число потоков для параллельного алгоритма;
    \item на выходе должны выдаваться вычисленные значения определенного
          интеграла каждым из алгоритмов, причем результаты должны совпадать;
    \item интегрируемые функции могут быть предложены пользователю на выбор.
\end{itemize}
