\chapter{Аналитическая часть}

В данном разделе представлено теоретическое описание алгоритмов сортировки вставками, перемешиванием и выбором.

\section{Алгоритм сортировки вставками}

В \textbf{алгоритме сортировки вставками} сортируемая последовательность
условно делится на входную неотсортированную часть и выходную отсортированную
часть. На каждом шаге из неотсортированной части выбирается элемент и
помещается на нужную позицию в уже отсортированной части. В начале алгоритма
считается, что первый элемент последовательности явлется отсортированной
частью, поэтому вставка элементов в отсортированную часть начинается со второго
элемента \cite{sorts}.

\section{Алгоритм сортировки перемешиванием}

\textbf{Алгоритм сортировки перемешиванем} является модификацией алгоритма
сортировки пузырьком. В отличие от сортировки пузырьком, где происходит обход
последовательности только в одном направлении, в алгоритме сортировки
перемешиванием после достижения одной из границ рабочей части
последовательности (то есть той части, которая еще не отсортирована и в которой
происходит смена элементов) меняет направление движения. При этом при движении
в одном направлении алгоритм перемещает к границе рабочей области максимальный
элемент, а в другом направлении -- минимальный элемент. Границы рабочей части
последовательности устанавливаются в месте последнего обмена \cite{sorts}. 

\section{Алгоритм сортировки выбором}

\textbf{Алгоритм сортировки выбором} в текущей последовательности находит
минимальный/максимальный элемент, производит обмен этого элемента со значением
первой неотсортированной позиции и сортирует оставшуюся часть
последовательности, исключив из рассмотрения уже отсортированные элементы
\cite{sorts}.

\section{Вывод}

В данном разделе были рассмотрены алгоритмы сортировки вставками,
перемешиванием и выбором. Из представленных описаний можно предъявить ряд
требований к разрабатываемому программному обеспечению:
\begin{itemize}[left=\parindent]
    \item на вход подается последовательность, которую необходимо
          отсортировать;
    \item на выходе должна выдаваться отсортированная последовательность;
    \item элементы последовательности должны быть представимы любым базовым
        типом, для которого определена операция сравнения.
\end{itemize}
