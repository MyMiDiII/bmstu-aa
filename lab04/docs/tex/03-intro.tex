\chapter*{Введение}
\addcontentsline{toc}{chapter}{Введение}

Сегодня программирование используется во многих научных и социальных областях.
Компьютерам требуется производить все более трудоемкие вычисления на больших
объемах данных. При этом предъявляются требования к скорости вычислений.

Одним из возможных решений увеличения производительности комьпьютера, то есть
скорости решения задач является параллельное программирование. На одном
устройстве паралельные вычисления можно организовать с помощью
\textbf{многопоточности} -- способности цетрального процессора или одного ядра
в многоядерном процессоре одновременно выполнять несколько процессов или
потоков, соответствующим образом поддерживаемых операционной системой. При
последовательной реализации какого-либо алгоритма, его программу выполняет
только одно ядро процессора. Если же реализовать алгоритм так, что независимые
вычислительные задачи смогут выполнять несколько ядер параллельно, то это
приведет к ускорению решения всей задачи в целом\cite{intro}.

Для реализации паралельных вычислений требуется выделить те участки алгоритма,
которые могут выполняться паралльно без изменения итогового резульатат, 
также необходми правильно организовать работу с данными, чтобы не потерять
вычисленные значения.

Одной из распространненых задач, решение которой используется в различных
областях, является численное итегрирование, поэтому в данной лабораторной
работе будет предпринята попытка ускорить вычисления определенных интегралов.

\textbf{Целью данной работы} является получение навыков
параллельного программирования на основе алгоритма
численного интегрирования методом средних прямоугольников.

Для достижения поставленной цели необходимо выполнить следующие
\textbf{задачи}:
\begin{itemize}[left=\parindent]
    \item изучить метод средних прямоугольников для численного интегрирования;
    \item описать возможности распараллеливания данного алгоритма;
    \item разработать последовательный и параллельный алгоритмы;
    \item реализовать каждый алгоритм;
    \item провести тестирование реализованных алгоритмов;
    \item провести сравнительный анализ алгоритмов по времени работы
          реализаций;
    \item сделать выводы по полученным результатам.
\end{itemize}
