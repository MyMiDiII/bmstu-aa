\chapter{Конструкторская часть}

В данном разделе разрабатываются последовательный и параллельный алгоритмы,
численного интегрирования методом средних прямоугольников, структура программы
и способы её тестироваия.

\section{Разработка алгоритмов}

На рисунках \ref{scheme:s01}-\ref{scheme:s02} представлена схема алгоритма
последовательного алгоритма численного интегрирования методом средних
прямоугольников с заданной точностью, на рисунках
\ref{scheme:p01}-\ref{scheme:p03} --- схема параллельного алгоритма.

\noindent
\scheme{90mm}{s01}{Схема последовательного алгоритма численного интегрирования
с заданной точностью}{s01}
\noindent
\scheme{90mm}{s02}{Схема последовательного алгоритма численного интегрирования
методом средних прямоугольников при заданном количестве участков
разбиения}{s02}
\noindent
\scheme{90mm}{p01}{Схема параллельного алгоритма численного интегрирования
с заданной точностью}{p01}
\noindent
\scheme{90mm}{p02}{Создание потоков для параллельного алгоритма численного
интегрирования}{p02}
\noindent
\scheme{90mm}{p03}{Схема параллельного алгоритма численного интегрирования
методом средних прямоугольников при заданном количестве участков
разбиения}{p03}

\section{Структура разрабатываемого ПО}

Для реализации разрабатываемого программного обеспечения будет использоваться
метод структурного программирования. Каждый из алгоритмов будет представлен
отдельной функцией, при необходимости будут выделены подпрограммы для каждой из
них. Также будут реализованы функции для ввода-вывода и функция, вызывающая все
подпрограммы для связности и полноценности программы.

\section{Классы эквивалентности при тестировании}

Для тестирования программного обеспечения во множестве тестов будут выделены
следующие классы эквивалентности:
\begin{itemize}[left=\parindent]
    \item совпадение верхнего и нижнего пределов интегрирования; 
    \item положительные пределы интегрирования;
    \item верхний предел интегрирования меньше нижнего предела;
    \item произвольные пределы интегрирования.
\end{itemize}

\section{Вывод}

В данном разделе были разработаны последовательный и параллельный алгоритмы,
была описана структура разрабатываемого ПО.  Для дальнейшей проверки
правильности работы программы были выделены классы эквивалентности тестов.
