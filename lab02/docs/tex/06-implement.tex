\chapter{Технологическая часть}

В данном разделе описаны требования к программному обеспечению, средства
реализации, приведены листинги кода и данные, на которых будет проводиться
тестирование.

\section{Требования к программному обеспечению}

Программа должна предоставлять следующие возможности:
\begin{itemize}[left=\parindent]
    \item выбор режима работы: для единичного эксперимента и для массовых
          эксперименов;
    \item в режиме единичного эксперимента ввод размеров и содержимого каждой
          матрицы и вывод полученного разными алгоритмами произведений;
    \item в режиме массовых экспериментов измерение времени при различных
          размерах матриц и построение графиков по полученным данным.
\end{itemize}

\section{Средства реализации}

Для реализации данной лабораторной работы выбран интерпретируемый язык
программирования высокого уровня Python\cite{python}, так как он позволяет
реализовывать сложные задачи за кратчайшие сроки за счет простоты синтаксиса и
наличия большого количества подключаемых библиотек. 

В качестве среды разработки выбран текстовый редактор Vim\cite{vim} c
установленными плагинами автодополнения и поиска ошибок в процессе написания,
так как он реализует быстрое перемещение по тексту программы и простое
взаимодействие с командной строкой.

Замеры времени проводились при помощи функции process\_time\_ns из библиотеки
time\cite{time}.

\section{Листинги кода}

В данном подразделе представлены листинги кода ранее описанных алгоритмов:
\begin{itemize}[left=\parindent]
    \item стандартный алгоритм умножения матриц (листинг \ref{lst:std});
    \item алгоритм Винограда (листинги \ref{lst:WinMulH}-\ref{lst:Winograd});
    \item оптимизированный алгоритм Винограда (листинги
          \ref{lst:optWinMulH}-\ref{lst:optWinograd}).
\end{itemize}

\noindent
\begin{minipage}{\linewidth}
\begin{lstinputlisting}[
	caption={Реализация стандартного алгоритма умножения матриц},
	label={lst:std},
	linerange={19-31}
]{../../src/matrixes.py}
\end{lstinputlisting}
\end{minipage}

\noindent
\begin{minipage}{\linewidth}
\begin{lstinputlisting}[
    caption={Реализация подпрограммы алгоритма Винограда для расчета значений
             массива mulH},
	label={lst:WinMulH},
    linerange={34-44}
]{../../src/matrixes.py}
\end{lstinputlisting}
\end{minipage}

\noindent
\begin{minipage}{\linewidth}
\begin{lstinputlisting}[
    caption={Реализация подпрограммы алгоритма Винограда для расчета значений
             массива mulV},
	label={lst:WinMulV},
    linerange={47-57}
]{../../src/matrixes.py}
\end{lstinputlisting}
\end{minipage}
\\
\\
\noindent
\begin{minipage}{\linewidth}
\begin{lstinputlisting}[
	caption={Реализация алгоритма Винограда},
	label={lst:Winograd},
    linerange={60-82}
]{../../src/matrixes.py}
\end{lstinputlisting}
\end{minipage}
\\
\\
\noindent
\begin{minipage}{\linewidth}
\begin{lstinputlisting}[
    caption={Реализация подпрограммы оптимизированного алгоритма Винограда для
    расчета значений массива mulH},
	label={lst:optWinMulH},
    linerange={85-95}
]{../../src/matrixes.py}
\end{lstinputlisting}
\end{minipage}

\noindent
\begin{minipage}{\linewidth}
\begin{lstinputlisting}[
    caption={Реализация подпрограммы оптимизированного алгоритма Винограда для
    расчета значений массива mulH},
	label={lst:optWinMulV},
    linerange={98-108}
]{../../src/matrixes.py}
\end{lstinputlisting}
\end{minipage}

\noindent
\begin{minipage}{\linewidth}
\begin{lstinputlisting}[
	caption={Реализация оптимизированного алгоритма Винограда},
	label={lst:optWinograd},
    linerange={111-134}
]{../../src/matrixes.py}
\end{lstinputlisting}
\end{minipage}

\section{Описание тестирования}

В таблице \ref{tab:tests} приведены функциональные тесты для алгоритмов
умножения матриц.

\begin{table}[h!]
	\begin{center}
        \caption{\label{tab:tests}Функциональные тесты}
        \begin{tabular}{|c|c|c|}
			\hline
            \textbf{Матрица 1} & \textbf{Матрица 2} &\textbf{Ожидаемый
            результат} \\ [2mm]
            \hline
			$\begin{pmatrix}
                1 & 2 & 3
			\end{pmatrix}$ &
			$\begin{pmatrix}
                1 & 2
			\end{pmatrix}$ &
			Не могут быть перемножены\\
            \hline
			$\begin{pmatrix}
                1 & 2 & 3\\
                1 & 2 & 3\\
                1 & 2 & 3
			\end{pmatrix}$ &
			$\begin{pmatrix}
                -1 & 2 & 3\\
                1 & -2 & 3\\
                1 & 2 & -3
			\end{pmatrix}$ &
			$\begin{pmatrix}
                4 & 4 & 0\\
                4 & 4 & 0\\
                4 & 4 & 0
			\end{pmatrix}$ \\
            \hline
			$\begin{pmatrix}
                1 & 2 & 3\\
                1 & 2 & 3
			\end{pmatrix}$ &
			$\begin{pmatrix}
                -1 & -3\\
                -2 & -2\\
                -3 & -1
			\end{pmatrix}$ &
			$\begin{pmatrix}
                -14 & -10\\
                -14 & -10
			\end{pmatrix}$ \\
            \hline
			$\begin{pmatrix}
                3 & -5\\
                1 & -2
			\end{pmatrix}$ &
			$\begin{pmatrix}
                2 & -5\\
                1 & -3
			\end{pmatrix}$ &
			$\begin{pmatrix}
			    1 & 0\\
                0 & 1
			\end{pmatrix}$ \\
            \hline
			$\begin{pmatrix}
                1 & 2 & 3\\
                1 & 2 & 3\\
                1 & 2 & 3
			\end{pmatrix}$ &
			$\begin{pmatrix}
                0 & 0 & 0\\
                0 & 0 & 0\\
                0 & 0 & 0
			\end{pmatrix}$ &
			$\begin{pmatrix}
                0 & 0 & 0\\
                0 & 0 & 0\\
                0 & 0 & 0
			\end{pmatrix}$ \\
            \hline
			$\begin{pmatrix}
                1 & 2 & 3\\
                1 & 2 & 3\\
                1 & 2 & 3
			\end{pmatrix}$ &
			$\begin{pmatrix}
                1 & 0 & 0\\
                0 & 1 & 0\\
                0 & 0 & 1
			\end{pmatrix}$ &
			$\begin{pmatrix}
                1 & 2 & 3\\
                1 & 2 & 3\\
                1 & 2 & 3
			\end{pmatrix}$ \\
            \hline
		\end{tabular}
	\end{center}
\end{table}

\section{Вывод}

В данном разделе были реализованы алгоритмы умножения матриц: стандартный, Винограда и оптимизированный Винограда. Также были написаны тесты для каждого класса эквивалентности, описанного в конструкторском разделе.
