\chapter{Технологическая часть}

В данном разделе описаны требования к программному обеспечению, средства
реализации, приведены листинги кода и данные, на которых будет проводиться
тестирование.

\section{Требования к программному обеспечению}

Программа должна предоставлять следующие возможности:
\begin{itemize}[left=\parindent]
    \item выбор режима работы: для единичного эксперимента и для массовых
          эксперименов;
    \item в режиме единичного эксперимента ввод размеров и содержимого каждой
          матрицы и вывод полученного разными алгоритмами произведений;
    \item в режиме массовых экспериментов измерение времени при различных
          размерах матриц и построение графиков по полученным данным.
\end{itemize}

\section{Средства реализации}

Для реализации данной лабораторной работы выбран интерпретируемый язык
программирования высокого уровня Python\cite{python}, так как он позволяет
реализовывать сложные задачи за кратчайшие сроки за счет простоты синтаксиса и
наличия большого количества подключаемых библиотек. 

В качестве среды разработки выбран текстовый редактор Vim\cite{vim} c
установленными плагинами автодополнения и поиска ошибок в процессе написания,
так как он реализует быстрое перемещение по тексту программы и простое
взаимодействие с командной строкой.

Замеры времени проводились при помощи функции process\_time\_ns из библиотеки
time\cite{time}.

\section{Листинги кода}

В данном подразделе представлены листинги кода ранее описанных алгоритмов:
\begin{itemize}[left=\parindent]
    \item (листинг \ref{lst:});
    \item (листинг \ref{lst:}-\ref{lst:});
    \item (листинг \ref{lst:}, \ref{lst:});
    \item (листинг \ref{lst:}).
\end{itemize}

% \noindent
% \begin{minipage}{\linewidth}
% \begin{lstinputlisting}[
% 	caption={},
% 	label={lst:},
% 	linerange={}
% ]{../../src/}
% \end{lstinputlisting}
% \end{minipage}

\section{Описание тестирования}

В таблице \ref{tab:tests} приведены функциональные тесты для алгоритмов

% \begin{table}[h]
% 	\begin{center}
% 		\caption{\label{tab:tests}Функциональные тесты}
% 		\begin{tabular}{|c|c|c|c|}
% 			\hline
% 			& & \multicolumn{2}{c|}{\bfseries Ожидаемый результат}\\ \cline{3-4}
% 			\bfseries Строка 1  & \bfseries Строка 2 &
%             \bfseries Левенштейн & \bfseries Дамерау-Левенштейн
% 			\csvreader{../data/csv/tests.csv}{}
% 			{\\\hline \csvcoli&\csvcolii&\csvcoliii&\csvcoliv}
% 			\\\hline
% 		\end{tabular}
% 	\end{center}
% \end{table}

\clearpage

\section{Вывод}

