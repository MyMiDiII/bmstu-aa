\chapter{Конструкторская часть}

В данном разделе разрабатываются алгоритмы умножения матриц, структура
программы и способы её тестирования, также проводится оценка трудоемкости
каждого из алгоритмов и оптимизация алгоритма Винограда.

\section{Разработка алгоритмов}

В данном подразделе приводятся схемы разработанных алгоритмов, оценка их
тредоемкости, на основе которой производится оптимизация алгоритма Винограда с
последующим описаним алгоритма в виде схемы.

\subsection{Схемы стандартного алгоритма умножения матриц и алгоритма
            Винограда}

На рисунке \ref{scheme:standart} приведена схема стандартного алгоритма умножения
матриц.

\noindent
\scheme{110mm}{standart}{Схема стандартного алгоритма умножения
матриц}{standart}

На рисунках \ref{scheme:Winograd}-\ref{scheme:countMulV} приведена схема
алгоритма Винограда.

\noindent
\scheme{225mm}{Winograd}{Схема алгоритма Винограда умножения матриц}{Winograd}
\noindent
\scheme{105mm}{countMulH}{Схема алгоритма вычисления сумм произведений пар
соседних элементов строк матрицы}{countMulH}
\noindent
\scheme{105mm}{countMulV}{Схема алгоритма вычисления сумм произведений пар
соседних элементов столбцов матрицы}{countMulV}

\section{Оценка трудоемкости алгоритмов}

\subsection{Модель вычислений}

\subsection{Трудоемкость стандартного алгоритма умножения матриц}

\subsection{Трудоемкость алгоритма умножения матриц Винограда}

\section{Оптимизация алгоритма Винограда}

Здесь про типы оптимизации

\subsection{Схема оптимизированного алгоритма Винограда}

На рисунке \ref{img:optWin} приведена схема оптимизированного алгоритма Винограда.

\subsection{Трудоемкость оптимизированного алгоритма Винограда}

\section{Структура разрабатываемого ПО}

Для реализации разрабатываемого программного обеспечения будет использоваться
метод структурного программирования. Каждый из алгоритмов будет представлен
отдельной функцией, при необходимости будут выделены подпрограммы для какждой
из функций. Также будут реализованы функции для ввода-вывода
и функция, вызывающая все подпрограммы для связности и полноценности
программы.

\textbf{ЗДЕСЬ МБ IDEF0}

\section{Классы эквивалентности при тестировании}

Для тестирования программного обеспечения во множестве тестов будут выделены
следующие классы эквивалентности:
\begin{itemize}[left=\parindent]
    \item
    \item
    \item
    \item
    \item
    \item
    \item
    \item
    \item
\end{itemize}

\section{Вывод}

