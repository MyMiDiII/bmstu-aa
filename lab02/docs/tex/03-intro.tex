\chapter*{Введение}
\addcontentsline{toc}{chapter}{Введение}

Умножение матриц является основным инструментом линейной алгебры и имеет
многочисленные применения в математике, физике, программировании
\cite{haskell}. При этом сложность стандартного алгоритма умножения матриц
$N \times N$ составляет $O(N^3)$ \cite{till}, что послужило причиной разработки
новых алгоритмов меньшей сложности. Одним из них является алгоритм Винограда с
асимптотической сложностью $O(N^{2.3755})$ \cite{haskell}.

\textbf{Целью данной работы} является изучение алгоритмов умножения матриц:
стандартного и Винограда, --- а также получение навыков расчета сложности
алгоритмов и их оптимизации.

Для достижения поставленной цели необходимо выполнить следующие
\textbf{задачи}:
\begin{itemize}[left=\parindent]
    \item изучить алгоритмы умножения матриц: стандартный и алгоритм Винограда;
    \item разработать каждый из алгоритмов;
    \item дать теоретическую оценку трудоемкости стандартного алгоритма и
          алгоритма Винограда;
    \item оптимизировать алгоритм Винограда и дать теоретическую оценку
          его трудоемкости;
    \item реализовать каждый из трех алгоритмов;
    \item провести тестирование реализованных алгоритмов;
    \item провести сравнительный анализ алгоритмов по процессорному времени
          работы реализации.
\end{itemize}
