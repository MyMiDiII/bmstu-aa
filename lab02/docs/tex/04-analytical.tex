\chapter{Аналитическая часть}

В данном разделе представлено теоретическое описание стандартного алгоритма
умножения матриц и алгоритма Винограда.

\section{Стандартный алгоритм умножения матриц}

Стандартный алгоритм умножения матриц является реализацией математического
определения произведения матриц, которое формулируется следующим образом:

    пусть даны матрица $A = (a_{ij})_{n \times p}$, имеющая $n$ строк и $p$
столбцов, и матрица ~$B = (b_{ij})_{p \times m}$, ~ имеющая ~ $p$ ~ строк ~ и ~
$m$ столбцов, ~ тогда ~ матрица ${C=(c_{ij})_{n \times m}}$, имеющая $n$
строк и $m$ столбцов, где:
\begin{equation}\label{eq:11}
    \begin{gathered}
        c_{ij}=a_{i1}b_{1j}+a_{i2}b_{2j} +...+a_{ip}b_{pj}=\sum\limits_{k=1}^p
        a_{ik}b_{kj};\\
        i = \overline{1,n};~j = \overline{1,m};
    \end{gathered}
\end{equation}
--- называется \textbf{произведением} матриц $A$ и $B$ \cite{math}.

\section{Алгоритм умножения матриц Винограда}

Из формулы \ref{eq:11} видно, что каждый элемент итоговой матрицы представляет
собой скалярное произведение соответсвующих строки и столбца исходных матриц,
которое, в свою очередь, допускает предварительную обработку, т.~е. часть
вычислений можно выполнить заранее.

Пусть даны два вектора $V=(v_1, v_2, v_3, v_4)$ и
$W=(w_1,w_2,w_3,w_4)$. Их скалярное произведение
представлено формулой \ref{eq:12}:
\begin{equation}\label{eq:12}
    V \cdot W = v_1w_1+v_2w_2+v_3w_3+v_4w_4
\end{equation}

Равенство \ref{eq:12} можно представить в виде:
\begin{equation}\label{eq:13}
    V \cdot W = (v_1+w_2)(v_2+w_1)+(v_3+w_4)(v_4+w_3)-v_1v_2-v_3v_4-w_1w_2-w_3w_4
\end{equation}

Хотя в выражении \ref{eq:13} больше операций, чем в выражении \ref{eq:12}:
девять сложений против трех, и шесть умножений против четырех, --- последнее
выражение допускает предварительную обработку, а именно операции вида
$v_iv_{i+1}$ и $w_iw_{i+1}$ для $i \in \overline{0,2..p}$ могут быть вычислены
заранее. Это позволить нам для каждого элемента выполнять только операции для
двух первых слагаемых в выражении \ref{eq:13}, то есть два умножения и пять
сложений, а также две операции сложения для учета уже вычисленных значений
\cite{winograd} .

\section{Вывод}

В данном разделе были рассмотрены два алгоритма умножения матриц: стандартный
и Винограда. Из представленных описаний можно предъявить ряд требований для
разрабатываемого программного опеспечения:
\begin{itemize}[left=\parindent]
    \item на вход должны подаваться две матрицы;
    \item на выходе должна выдаваться результирующая матрица, являющаяся
          произведением двух исходных;
    \item так как произведение определено не для всех пар матриц, а только для
          таких, у которых в паре количество столбцов в первой матрице равно
          количеству строк во второй, программа должна корректно реагировать на
          недопустимые входные размеры матриц.
\end{itemize}
