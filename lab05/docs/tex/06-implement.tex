\chapter{Технологическая часть}

В данном разделе описаны требования к программному обеспечению, средства
реализации, приведены листинги кода и данные, на которых будет проводиться
тестирование.

\section{Требования к программному обеспечению}

Программа должна предоставлять следующие возможности:
\begin{itemize}[left=\parindent]
    \item выбор режима работы: для единичного эксперимента и для массовых
          эксперименов;
    \item в режиме единичного эксперимента выбор функции для интегрирования,
          ввод пределов интегрирования, точности и числа потоков для
          параллельной реализации;
    \item в режиме массовых экспериментов измерение времени работы каждого из
          алгоритмов в зависимости от точности и числа потоков.
\end{itemize}

\section{Средства реализации}

Для реализации данной лабораторной работы выбран компилируемый язык
програмирования \texttt{C++}\cite{cpp}, так как он предоставляет необходимые библиотеки
для работы с потоками. Интерпретируемый язык программирования высокого уровня
Python\cite{python} был выбран для визуализации данных эксперимента, так как он
предоставляет большое число настроек параметров графика с использованием
простого синтаксиса. 

В качестве среды разработки выбран текстовый редактор Vim\cite{vim} c
установленными плагинами автодополнения и поиска ошибок в процессе написания,
так как он реализует быстрое перемещение по тексту программы и простое
взаимодействие с командной строкой.

Замеры~~~~~~времени~~~~~~проводились~~~~~~при~~~~~~помощи~~~~~~функции \texttt{std::chrono::system\_clock::now()}
из библиотеки chrono\cite{chrono}.

\section{Листинги кода}

В данном подразделе представлены листинги кода ранее описанных алгоритмов:
\begin{itemize}[left=\parindent]
    \item последовательный алгоритм численного итегрирования методом средних
        прямоугольников с заданной точностью (листинги
        \ref{lst:smidpoint}-\ref{lst:sbyprec});
    \item параллельный алгоритм численного итегрирования методом средних
        прямоугольников с заданной точностью (листинг
        \ref{lst:pmidpoint}-\ref{lst:pbyprec}).
\end{itemize}

%\mylisting{Функция интегрирования методом средних прямоугольников с заданным
%числом участков разбиения (последовательный
%алгоритм)}{smidpoint}{12-25}{integral.cpp}
%
%\mylisting{Функция вычисления интеграла с заданной точностью (последовательный
%алгоритм)}{sbyprec}{27-42}{integral.cpp}
%
%\mylisting{Функция интегрирования методом средних прямоугольников с заданным
%числом участков разбиения (параллельный
%алгоритм)}{pmidpoint}{44-61}{integral.cpp}
%
%\mylisting{Функция создания потоков для интегрирования методом средних
%прямоугольников}{multhreads}{62-83}{integral.cpp}
%
%\mylisting{Функция вычисления интеграла с заданной точностью (параллельный
%алгоритм)}{pbyprec}{27-42}{integral.cpp}

\section{Описание тестирования}

В таблице \ref{tab:tests} приведены функциональные тесты для алгоритмов
интегрирования на функции $f(x) = x^2$.

\begin{table}[h!]
	\begin{center}
    \begin{threeparttable}
        \captionsetup{justification=raggedright,singlelinecheck=off}
        \caption{\label{tab:tests}Функциональные тесты}
        \begin{tabular}{|c|c|}
			\hline
            \textbf{Пределы интегрирования} & \textbf{Ожидаемый результат} \\ [2mm]
            \hline
            0 0
            &
            0
            \\
            \hline
            0 1
            &
            0.3333
            \\
            \hline
            1 0
            &
            -0.3333
            \\
            \hline
            -1 1
            &
            0.6666
            \\
            \hline
		\end{tabular}
    \end{threeparttable} 
	\end{center}
\end{table}

\section{Вывод}

В данном разделе были реализованы последовательный и параллельный алгоритмы
численного интегрирования методом средних прямоугольников. Также были написаны
тесты для каждого класса эквивлентности, описанного в конструкторском разделе.
