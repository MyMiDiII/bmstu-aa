\chapter{Аналитическая часть}

В данном разделе представлено теоретическое описание конвейерной обработки
данных и алгоритмов каждой стадии обработки.

\section{Конвейерная обработка данных}

\textbf{Конвейерная обработка данных} -- это подход, позволяющий созадавать
эффективные параллельные реализации обычных неспециализированных алгоритмовс
\cite{Pogorelov}. Общая идея конвейера связана с разбиением некоторого
процесса обработки объектов на несколько независимых этапов и организацией
одновременного (параллельного) выполнения этих этапов обработки различных
объектов, передвигающихся по конвейеру от одного этапа к другому.  При движении
объектов по конвейеру на разных его участках выполняются разные операции, а при
достижении каждым объектом конца конвейера он окажется полностью обработанным
\cite{Bogoslovskiy}.

При реализации конвейерной обработки каждый из этапов должен \cite{Pogorelov}:

\begin{enumerate}
    \item получить данные;
    \item обработать данные;
    \item передать данные следующим этапам.
\end{enumerate}

В данной лабораторной работе будет реализован конвейер, состоящий из трех
этапов обработки данных, каждый из которых будет реализован в отдельном потоке.
Передача данных между потоками будет реализована с помощью четырех очередей:
\begin{itemize}
    \item от главного потока к первому этапу;
    \item от первого этапа ко второму этапу;
    \item от второго этапа к третьему этапу;
    \item от третьего этапа к главному потоку.
\end{itemize}

\section{Стадии конвейерной обработки}

Для проведения исследования в данной лабораторной работе на конвейере будут
генерироваться зашифрованные сообщения. На первой ленте будет генерироваться
случайное сообщение на английском языке. На второй ленте каждое слово в
сообщении будет записываться в обратном порядке. На третьей ленте к сообщению
будет применен шифр Веженера.

Далее описываются алгоритмы, выполняющиеся на каждой ленте конвейера.

\subsection{Генерация сообщения}

На первой ленте генерируются случайные сообщения на английском языке и
переводятся в нижний регистр для последующей неразличимости первых слов
предложений и имен собственных.

\subsection{Перестановка букв в словах в обратном порядке}

На второй ленте у каждого слова сгенерированного на первом этапе сообщения
переставляются буквы в обратном порядке.

\subsection{Шифр Веженера}

На третьей ленте полученное на втором этапе сообщение зашифровывается с
помощью шифра Веженера.

Шифр Веженера является частным случаем многоалфавитной замены. Формально
данный алгоритм шифрования можно описать следующим образом. Выбирается ключ
шифрования -- набор из $m$ целых чисел $k = (k_1, k_2, ..., k_m)$ (обычно
выбирается какое-либо слово, буквам которого ставится в соответствие число --
порядковый номер в алфавите) \cite{Vegenere}. Каждый символ зашифрованного
сообщения $c_i$ получается из соответствующего символа $t_i$ данного текста по
следующей формуле:

\begin{equation}
    c_i = t_i + k_{i \mod m + 1} \pmod {|V|},
\end{equation}

где $i \in \{1, 2, ... n\}$ ($n$ -- длина текста), $|V|$ -- мощность алфавита.

\section{Вывод}

В данном разделе была описана модель конвейерной обработки данных, также были
представлены описания алгоритмов, выполняющихся на каждой ленте конвейера. В
качестве реализуемого поэтапного алгоритма была выбрана генерация зашифрованных
сообщений.  Из представленных описаний можно предъявить ряд требований к
разрабатываемому программному обеспечению:
\begin{itemize}[left=\parindent]
    \item на вход должно подаваться количество обрабатываемых заявок;
    \item при неверных входных данных должно выдаваться сообщение об ошибке;
    \item на выходе должны выдаваться логи работы каждого из методов обработки
        (последовательного и конвейерного).
\end{itemize}
