\chapter{Аналитическая часть}

В данном разделе представлено теоретическое описание алгоритмов численного
интегрирования методом средних прямоугольников.

\section{Конвейерная обработка данных}

\section{Генерация зашифрованных сообщений}

Для проведения исследования в данной лабораторной работе на конвейере будут генерироваться зашифрованные сообщения. На первой ленте будет генерироваться случайное сообщение на английском языке. На второй ленте каждое слово в сообщении будет записываться в обратном порядке. На третьей ленте к сообщению будет применен шифр Веженера.

\subsection{Шифр Веженера}

Шифр Веженера является частным случаем многоалфавитной замены. Формально даннный алгоритм шифрования можно описать следующим образом. Выбирается ключ шифравания 

http://window.edu.ru/resource/256/20256/files/rsu572.pdf

\section{Вывод}

В данном разделе был рассмотрен алгоритм численного интегрирования методом
средних прямоугольнико, так же был описан механизм распараллеливания данного
алгоритма. Из представленных описаний можно предъявить ряд требований к
разрабатываемому программному обеспечению:
\begin{itemize}[left=\parindent]
    \item на вход должны подаваться пределы интегрирования, заданная точность,
          а также число потоков для параллельного алгоритма;
    \item на выходе должны выдаваться вычисленные значения определенного
          интеграла каждым из алгоритмов, причем результаты должны совпадать;
    \item интегрируемые функции могут быть предложены пользователю на выбор.
\end{itemize}
