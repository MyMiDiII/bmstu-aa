\chapter*{Заключение}
\addcontentsline{toc}{chapter}{Заключение}

В ходе исследования был проведен сравнительный аналализ последовательной и
конвейерной реализации алгорима поэтапного шифрования сообщений. В результате
исследования было выяснено, что при количестве заявок более 10 конвейерная
реализация дает выигрыш в $1.7$ раза по сравнению с последовательной, что 
говорит о преимуществе параллельной реализации этапов конвейера при решении
поставленной задачи.

В ходе выполения лабораторной работы:
\begin{itemize}[left=\parindent]
    \item были описаны и разработаны алгоритмы этапов конвейерной обработки данных;
    \item были описаны и разработаны линейная и конвейерная обрабоки данных;
    \item был реализован каждый из описанных алгоритмов;
    \item по экспериментальным данным были сделаны выводы об эффективности по
          времени каждого из реализованных алгоритмов;
    \item были получены зависимости времени работы линейной и конвейерной реализаций от числа и размера заявок.
\end{itemize}

Таким образом, все поставленные задачи были выполнены, а цель достигнута.
